\documentclass[
	letterpaper,
	10pt,
	twoside
]{CookBook}

\begin{document}

\makecoverpage{
	title={My Cookbook},
	subtitle={Our Favorite Recipes},
	author={Kevin Nowaczyk},
	titlefontsize={\fontsize{36pt}{38pt}},
	subtitlefontsize={\fontsize{24pt}{26pt}},
	image={../images/book/cover.jpg},
	opacity={0.6},
	bgcolor={darkgrey},
	textcolor={white},
	shadowoffset={0.05cm}
}

%----------------------------------------------------------------------------------------
%	PREFACE
%----------------------------------------------------------------------------------------

\makeprefacepage{
	title={Preface},
	text={Welcome to this collection of recipes, gathered from three generations of family cooking and culinary adventures around the world. Each recipe tells a story—some passed down through handwritten notes on yellowed paper, others discovered during travels through bustling markets and quiet countryside kitchens.\newline

	This cookbook is more than just a collection of instructions and ingredients. It's a celebration of the joy that comes from creating something delicious, the warmth of sharing a meal with loved ones, and the memories that form around the dinner table. From simple weekday breakfasts to elaborate weekend feasts, these recipes have been tested, adjusted, and perfected over countless meals.\newline

	Whether you're a beginner just learning your way around the kitchen or an experienced cook looking for new inspiration, I hope these recipes bring as much happiness to your table as they have brought to ours. Don't be afraid to make them your own—the best recipes are the ones that evolve with each telling.\newline

	Happy cooking!},
	layout={single},
	image={../images/book/preface.jpg}
}

%----------------------------------------------------------------------------------------
%	TABLE OF CONTENTS
%----------------------------------------------------------------------------------------

\maketoc

%----------------------------------------------------------------------------------------
%	BREAKFAST CHAPTER PAGE
%----------------------------------------------------------------------------------------

\makechapterpage{
	title={Breakfast},
	bgcolor={paleorange},
	image={../images/book/breakfast.jpg},
	layout={right}
}


%----------------------------------------------------------------------------------------------------------


%----------------------------------------------------------------------------------------
%	RECIPE EXAMPLE - Breakfast Sausage Casserole
%----------------------------------------------------------------------------------------

\recipe{%
	layout={simple},
	imageheight={0.25\paperheight},
	imageoverlayspace={0.2\paperheight},
	title = {Breakfast Sausage Casserole},
	description = {This satisfying recipe is perfect to make for weekend guests. Assemble and refrigerate the casserole the night before, and just pop it in the oven the next morning. Look for turkey sausage near other breakfast-style sausage in the frozen foods section.},
	serves = {8},
	preptime = {15 mins},
	cookingtime = {65 mins},
	difficulty = {Beginner},
	origin = {USA},
	tags = {Breakfast},
	indexes = {Breakfast Casserole, Recipes!Breakfast, American cuisine},
	ingredients = {
	    \ingredient{1 (16-ounce) package sage sausage}
		},
	instructions = {
	    \instruction{Heat a large nonstick skillet over medium-high heat. Coat pan with cooking spray. Add sausage to pan, cook 5 minutes or until browned, stirring well to crumble.}
	}
}


%----------------------------------------------------------------------------------------------------------


%----------------------------------------------------------------------------------------
%	RECIPE EXAMPLE - Bacon, Ham, and Egg Hash (Columns Layout, No Image)
%----------------------------------------------------------------------------------------

\recipe{%
    image = {../images/recipes/hash.jpeg},
	title = {Bacon, Ham, and Egg Hash},
	indexes = {Hash, Recipes!Breakfast, Potato, Eggs, Quick meals},
	description = {Golden, crispy French toast with a custardy center. A weekend breakfast favorite that's easy to make yet feels special. Perfect with maple syrup and fresh berries.},
	serves = {4},
	preptime = {10 mins},
	cookingtime = {15 mins},
	difficulty = {Beginner},
	origin = {USA},
	tags = {Breakfast, Quick},
	ingredients = {
			\ingredientsection{Custard Mixture}
			\ingredient{4 large eggs}
			\ingredient{1 cup whole milk}
			\ingredient{2 tbsp granulated sugar}
			\ingredient{1 tsp vanilla extract}
			\ingredient{1 tsp ground cinnamon}
			\ingredient{¼ tsp ground nutmeg}
			\ingredient{Pinch of salt}

			\ingredientsection{Cooking and Serving}
			\ingredient{8 thick slices bread (brioche, challah, or white bread)\note{Day-old bread works best as it absorbs the custard mixture without falling apart. Brioche or challah bread creates the most luxurious French toast, but any thick-sliced bread will work.}}
			\ingredient{2-3 tbsp butter for cooking}
			\ingredient{Maple syrup, for serving}
			\ingredient{Fresh berries, for serving}
			\ingredient{Powdered sugar for dusting (optional)}
		},
	instructions ={
			\instruction{In a shallow bowl or baking dish, whisk together eggs, milk, sugar, vanilla extract, cinnamon, nutmeg, and salt until well combined.}
			\instruction{Heat a large skillet or griddle over medium heat and add 1 tablespoon of butter.}
			\instruction{Dip each bread slice into the custard mixture, allowing it to soak for about 5 seconds per side. Don't oversoak or the bread will fall apart.}
			\instruction{Place soaked bread slices in the heated skillet. Cook for 2-3 minutes per side until golden brown and crispy on the outside.}
			\instruction{Transfer cooked French toast to a plate and keep warm. Repeat with remaining bread slices, adding more butter to the pan as needed.}
			\instruction{Serve immediately topped with maple syrup, fresh berries, and a dusting of powdered sugar if desired.}
		}
}

\recipe{%
    %layout={simple},
    image={../images/recipes/gravy_pizza.png},
    columnratio={0.35, 0.65},
	title = {Gravy Pizza},
	indexes = {Pizza, Eggs, Recipes!Breakfast},
	description = {Recipe Description here.},
	serves = {12},
	preptime = {15 mins},
	cookingtime = {10 mins},
	difficulty = {Beginner},
	origin = {USA},
	tags = {Breakfast, Pizza, Quick, Baked},
	vegetarian = {yes},
	spicy = {yes},
	extrainstructioninfo = {Place additional comments here.},
	ingredients = {
	    \ingredientsection{Section 1}
	    \ingredient{First Ingredient}
	},
	instructions = {
	    \instructionsection{First Section}
	    \instruction{Place the text for the first step here}
	}
}

%----------------------------------------------------------------------------------------------------------


%----------------------------------------------------------------------------------------
%	APPETIZERS CHAPTER PAGE
%----------------------------------------------------------------------------------------

\makechapterpage{
	title={Appetizers},
	image={../images/book/salad.jpg}
}

%----------------------------------------------------------------------------------------------------------


%----------------------------------------------------------------------------------------
%	RECIPE - Esquites (Columns Layout)
%----------------------------------------------------------------------------------------

\recipe{%
    image = {../images/recipes/esquites.jpeg},
	title = {Mexican Street Corn (Esquites)},
	indexes = {Recipes!Appetizers, Corn, Vegetarian recipes, Mexican cuisine},
	description = {From SeriousEats},
	serves = {4},
	preptime = {5 mins},
	cookingtime = {15 mins},
	difficulty = {Beginner},
	origin = {Mexico},
	tags = {Mexican, Quick, Vegetarian},
	vegetarian = {yes},
	ingredients = {
			\ingredient{2 tablespoon vegtable oil}
			\ingredient{4 ears fresh corn (about 3 cups) or 2 cans}
			\ingredient{kosher salt}
			\ingredient{2 ounces fet or corija cheese, finely crumbled}
			\ingredient{½ cup finely sliced scallions}
			\ingredient{½ cup cilantro, finely chopped}
			\ingredient{1 jalapeño pepper, finely chopped }
			\ingredient{2 tablespoons mayonnaise}
			\ingredient{1 tablespoon fresh lime juice}
			\ingredient{chili powder to taste}
		},
	instructions ={
			\instruction{Heat oil in a large nonstick skillet or wok over high heat until shimmering. Add corn kernels, season to taste with salt, toss once or twice, and cook without moving until charred on one side, about 2 minutes. Toss corn, stir, and repeat until charred on second side, about 2 minutes longer. Continue tossing and charring until corn is well charred all over, about 10 minutes total. Transfer to a large bowl.}
			\instruction{Add cheese, scallions, cilantro, jalapeño, garlic, mayonnaise, lime juice, and chile powder and toss to combine. Taste and adjust seasoning with salt and more chile powder to taste. Serve immediately.}
		}
}

\recipe{%
    image = {../images/recipes/BG_pasta_salad.jpg},
	title = {Broccoli, Grape, and Pasta Salad},
	description = {If you're a broccoli salad fan, you'll love the combination of these colorful ingredients. Cook the pasta al dente so it's firm enough to hold its own when tossed with the tangy-sweet salad dressing.},
	ingredients = {
	    \ingredient{1 cup chopped pecans}
	},
	instructions = {
	    \instruction{Preheat oven to 350°. Bake pecans in a single layer in a shallow pan 5 to 7 minutes or until lightly toasted and fragrant, stirring halfway through.}
	}
}

\recipe{%
    image = {../images/recipes/potato_salad.jpg},
	title = {Roasted Potato Salad with Mustard Dressing},
	description = {This tangy side dish with sweet onions and honey pairs beautifully with burgers or steak. This salad is best chilled but can stay at room temperature for up to two hours. You can also use sweet-hot mustard in place of the Dijon and honey for a zesty flavor.},
	serves = {8},
	ingredients = {
	    \ingredient{3 pounds small red potatoes, cut into 1-inch pieces}
		\ingredient{}
	    \ingredient{}
	    \ingredient{}
	    \ingredient{}
	    \ingredient{}
	    \ingredient{}
	    \ingredient{}
	    \ingredient{}
	    \ingredient{}
	    \ingredient{}
	    \ingredient{}
	},
	instructions = {
	    \instruction{Preheat oven to 400°}
	}
}

\recipe{%
	title = {Sourdough Stuffing with Pears and Sausage},
	description = {Sourdough bread gives the stuffing a tangier flavor than French bread, but you can use the latter in a pinch.},
	serves = {12},
	ingredients = {
	    \ingredient{8 cups (½-inch) cubed sourdough bread (about 12 ounces)}
	},
	instructions = {
	    \instruction{Preheat oven to 425°}
	}
}

\recipe{%
    % layout={simple},
    % image={../images/recipes/foo.jpg},
    columnratio={0.35, 0.65},
	title = {Restaurant Style Mexican Rice},
	indexes = {Recipe Name, Key Ingredient, Recipes!Breakfast, Vegetarian recipes, French cuisine, Quick meals},
	description = {Recipe Description here.},
	serves = {12},
	preptime = {15 mins},
	cookingtime = {10 mins},
	difficulty = {Beginner},
	origin = {USA},
	tags = {Dessert, Cookies, Quick, Baked},
	vegetarian = {yes},
	spicy = {yes},
	extrainstructioninfo = {Place additional comments here.},
	ingredients = {
	    \ingredientsection{Section 1}
	    \ingredient{First Ingredient}
	},
	instructions = {
	    \instructionsection{First Section}
	    \instruction{Place the text for the first step here}
	}
}

\recipe{%
    % layout={simple},
    image={../images/recipes/rice_pilaf.jpg},
    columnratio={0.35, 0.65},
	title = {Rice Pilaf},
	indexes = {Rice Pilaf, Rice!Rice Pilaf},
	description = {From Alton Brown},
	serves = {6},
	preptime = {15 mins},
	cookingtime = {30 mins},
	difficulty = {Beginner},
	origin = {USA},
	tags = {Dessert, Cookies, Quick, Baked},
	vegetarian = {yes},
	spicy = {yes},
	extrainstructioninfo = {Place additional comments here.},
	ingredients = {
        \ingredient{1 tablespoon unsalted butter}
        \ingredient{1/2 medium onion, finely chopped}
        \ingredient{1/2 medium red bell pepper, finely chopped}
        \ingredient{1 1/2 teaspoons kosher salt plus 2 pinches}
        \ingredient{2 cups long-grain white rice}
        \ingredient{1 pinch saffron, steeped in 1/4 cup hot but not boiling water}
        \ingredient{2 1/2 cups chicken broth}
        \ingredient{1 1-by-2-inch strip orange zest}
        \ingredient{2 bay leaves}
        \ingredient{1/2 cup peas, fresh or frozen}
        \ingredient{1/4 cup golden raisins}
        \ingredient{1/4 cup pistachios, chopped}
	},
	instructions = {
	    \instruction{Preheat the oven to 350.}
	    \instruction{Melt the butter in a 3-quart saucier over medium heat.}
	    \instruction{Stir in the onion, bell pepper and 2 pinches of salt.}
	    \instruction{Decrease the heat to low and sweat until the onion is translucent and aromatic but not browned, 3 to 4 minutes. Increase the heat to medium and add the rice. Cook, stirring frequently, until you smell nuts, another 3 to 4 minutes.}
	    \instruction{Add the saffron and its water, the broth, orange zest, bay leaves and the remaining 1 1/2 teaspoons salt. Increase the heat and bring to a boil.}
	    \instruction{OK, now the weird part: Thoroughly wet a clean towel, kill the heat, scatter the peas on top of the rice, then place the towel across the top of the saucier. Top with the lid, then fold the towel corners up over the lid.}
	    \instruction{Transfer the saucier (towel and all) to the oven and bake 15 minutes.}
	    \instruction{Remove and rest at room temperature for 15 more minutes without opening the lid.}
	    \instruction{Fish out the orange zest and bay leaves. Turn the pilaf out onto a platter, fluff with a large fork and garnish with the raisins and pistachios. Serve family-style, right in the middle of the table.}
	}
}

%----------------------------------------------------------------------------------------
%	ENTREES CHAPTER PAGE
%----------------------------------------------------------------------------------------

\makechapterpage{
	title={Entrees},
	image={../images/book/pasta.jpg}
}

%----------------------------------------------------------------------------------------------------------


%----------------------------------------------------------------------------------------
%	RECIPE - Fajitas
%----------------------------------------------------------------------------------------

\recipe{%
	title = {Beef and Chicken Fajitas with Peppers and Onions},
	indexes = {Chicken!Beef and Chicken Fajitas with Peppers and Onions, Beef!Beef and Chicken Fajitas with Peppers and Onions, Tex-Mex!Beef and Chicken Fajitas with Peppers and Onions, Grilled!Beef and Chicken Fajitas with Peppers and Onions},
	description = {From Cooking Light Magazine.},
	serves = {4},
	preptime = {30 mins},
	cookingtime = {20 mins},
	difficulty = {Beginner},
	origin = {Tex-Mex},
	tags = {Chicken, Beef, Grilled, Tex-Mex, Maranade},
	ingredients = {
			\ingredientsection{Marinade}
			\ingredient{¼ cup olive oil}
			\ingredient{1 teaspoon lime rind,\\ grated}
			\ingredient{2½ tablespoons lime juice}
			\ingredient{2 tablespoons Worcestershire sauce}
			\ingredient{1½ teaspoons ground cumin}
			\ingredient{1 teaspoon salt}
			\ingredient{½ teaspoon dried oregano}
			\ingredient{½ teaspoon fresh coarse ground black pepper}
			\ingredient{3 garlic cloves,\\ minced}
			\ingredient{1 (14¼ ounce) can low sodium beef broth}

			\ingredientsection{Fajitas}
			\ingredient{1 (1-pound) flank steak}
			\ingredient{1 pound skinned, boned chicken breast}
			\ingredient{2 red bell peppers,\\ each cut into 12 wedges}
			\ingredient{2 green bell peppers,\\ each cut into 12 wedges}
			\ingredient{1 large  Vidalia onion,\\ cut into 16 wedges}
			\ingredient{cooking spray}
			\ingredient{16 (6-inch) flour tortillas}
			\ingredient{salsa}
			\ingredient{sour cream}
			\ingredient{cilantro}
			\ingredient{cheese}
			\ingredient{guacamole}
		},
	instructions = {
			\instructionsection{Preparing the Marinade}
			\instruction{Combine first 10 ingredients in a large bowl; set aside.}
			\instructionsection{Fajitas}
			\instruction{To prepare fajitas, trim fat from steak. Score a diamond pattern on both sides of the steak. Combine 1½ cups marinade, steak, and chicken in a large zip-top plastic bag. Seal and marinate in refrigerator 4 hours or overnight, turning occasionally. Combine remaining marinade, bell peppers, and onion in a zip-top plastic bag. Seal and marinate in refrigerator for 4 hours or overnight, turning occasionally.}
			\instruction{Prepare grill.}
			\instruction{Remove steak and chicken from bag; discard marinade. Remove vegetables from bag; reserve marinade. Place reserved marinade in a small saucepan; set aside. Place steak, chicken, and vegetables on grill rack coated with cooking spray; cook 8 minutes on each side or until desired degree of doneness.}
			\instruction{Wrap tortillas tightly in foil; place tortilla packet on grill rack the last 2 minutes of grilling time. Bring reserved marinade to a boil. Cut steak and chicken diagonally across the grain into thin slices. Place the steak, chicken, and vegetables on a serving platter; drizzle with reserved marinade.}
			\instruction{Arrange about 1 ounce steak, about 1 ounce chicken, 3 bell pepper wedges, and 1 onion wedge in a tortilla; top with 1 tablespoon salsa, about 1 teaspoon sour cream, and ½ tablespoon cilantro.
Fold sides of tortilla over filling. Garnish with cilantro sprigs, if desired. Serve immediately.}
		}
}

\recipe{%
    % layout={simple},
    % image={../images/recipes/foo.jpg},
    columnratio={0.35, 0.65},
	title = {Apple Cider-Brined Turkey with Savory Herb Gravy},
	indexes = {Pizza, Eggs, Recipes!Breakfast},
	description = {Recipe Description here.},
	serves = {12},
	preptime = {15 mins},
	cookingtime = {10 mins},
	difficulty = {Beginner},
	origin = {USA},
	tags = {Breakfast, Pizza, Quick, Baked},
	vegetarian = {yes},
	spicy = {yes},
	extrainstructioninfo = {Place additional comments here.},
	ingredients = {
	    \ingredientsection{Brine}
	    \ingredient{8 cups apple cider}
		\ingredient{⅔ cup kosher salt}
		\ingredient{⅔ cup sugar}
		\ingredient{1 tablespoon black peppercorns, coarsely crushed}
		\ingredient{}
		\ingredient{}
		\ingredient{}
		\ingredient{}
		\ingredient{}
		\ingredient{}
		\ingredient{}
		
		\ingredientsection{Brine}
        \ingredient{4 garlic cloves}
        \ingredient{4 sage leaves}
        \ingredient{4 thyme sprigs}
        \ingredient{4 parsley sprigs}
        \ingredient{1 onion, quartered}
        \ingredient{1 (14-ounce) can fat-free, less-sodium chicken broth}
        \ingredient{2 tablespoons unsalted butter, melted and divided}
        \ingredient{1 teaspoon freshly ground black pepper, divided}
        \ingredient{1/2 teaspoon salt, divided}

		\ingredientsection{Gravy}
		\ingredient{2 teaspoons vegetable oil}
        \ingredient{turkey neck and giblets}
        \ingredient{12 cups water}
        \ingredient{6 black peppercorns}
        \ingredient{4 parsley sprigs}
        \ingredient{2 thyme sprigs}
        \ingredient{1 yellow onion, unpeeled and quartered}
        \ingredient{1 carrot, cut into 2-inch pieces}
        \ingredient{1 celery stalk, cut into 2-inch pieces}
        \ingredient{1 bay leaf}
		\ingredient{reserved turkey drippings}
        \ingredient{3 tablespoons all-purpose flour}
        \ingredient{½ teaspoon salt}
        \ingredient{¼ teaspoon freshly ground black pepper}
	},
	instructions = {
	    \instructionsection{First Section}
	    \instruction{Place the text for the first step here}
	}
}

\recipe{%
    % layout={simple},
    image={../images/recipes/herbed_chicken_parmesan.jpg},
    columnratio={0.35, 0.65},
	title = {Herbed Chicken Parmesan},
	indexes = {Herbed Chicken Parmesan, Italian Cuisine, Chicken, Recipes!Entree},
	description = {This lightened version of an Italian favorite loses some of the fat but none of the flavor. Cooking Light.},
	serves = {4},
	preptime = {15 mins},
	cookingtime = {20 mins},
	difficulty = {Beginner},
	origin = {Italy},
	tags = {Italian, Quick, Baked},
	% vegetarian = {yes},
	% spicy = {yes},
	extrainstructioninfo = {Place additional comments here.},
	ingredients = {
	    \ingredient{⅓ cup (1 1/2 ounces) grated fresh Parmesan cheese, divided}
	    \ingredient{¼ cup dry breadcrumbs}
	    \ingredient{1 tablespoon minced fresh parsley}
	    \ingredient{½ teaspoon dried basil}
	    \ingredient{¼ teaspoon salt, divided}
	    \ingredient{1 large egg white, lightly beaten}
	    \ingredient{1 pound chicken breast tenders}
	    \ingredient{1 tablespoon butter}
	    \ingredient{1 ½ cups bottled fat-free tomato-basil pasta sauce (such as Muir Glen Organic)}
	    \ingredient{2 teaspoons balsamic vinegar}
	    \ingredient{¼ teaspoon black pepper}
	    \ingredient{⅓ cup (1 1/2 ounces) shredded provolone cheese}
		\ingredient{orzo}
		\ingredient(broccoli)
		\ingredient{lemon zest}
		\ingredient{garlic}
	},
	instructions = {
	    \instruction{Preheat broiler.}
	    \instruction{Combine 2 tablespoons of Parmesan, breadcrumbs, parsley, basil, and 1/8 teaspoon salt in a shallow dish. Place egg white in a shallow dish. Dip each chicken tender in egg white; dredge in the breadcrumb mixture. Melt butter in a large nonstick skillet over medium-high heat. Add chicken; cook 3 minutes on each side or until done. Set aside.}
	    \instruction{Combine 1/8 teaspoon salt, pasta sauce, vinegar, and pepper in a microwave-safe bowl. Cover with plastic wrap; vent. Microwave sauce mixture at HIGH 2 minutes or until thoroughly heated. Pour the sauce over chicken in pan. Sprinkle evenly with the remaining Parmesan and provolone cheese. Wrap handle of pan with foil, and broil 2 minutes or until the cheese melts.}
	}
}

\recipe{%
    % layout={simple},
    image={../images/recipes/pulled_chicken.jpg},
    columnratio={0.35, 0.65},
	title = {Pulled Chicken Sandwiches},
	indexes = {Sandwich, Chicken, Recipes!Breakfast},
	description = {Dinner guests are guaranteed to be impressed with this deceptively easy Pulled Chicken Sandwich recipe, which includes a seven-ingredient rub and a simple 15-minute sauce that comes together while the chicken grills. Serve on buns, over fresh greens, or on top of a baked potato for a filling dinner. Cooking Light},
	serves = {8},
	preptime = {20 mins},
	cookingtime = {20 mins},
	difficulty = {Beginner},
	origin = {USA},
	tags = {Sandwich, Chicken},
	% vegetarian = {yes},
	% spicy = {yes},
	extrainstructioninfo = {The chicken and sauce can be made up to two days ahead and stored in the refrigerator.},
	ingredients = {
	    \ingredientsection{Chicken}
	    \ingredient{2 tablespoons dark brown sugar}
	    \ingredient{1 teaspoon paprika}
	    \ingredient{¾ teaspoon ground cumin}
	    \ingredient{½ teaspoon ground chipotle chile pepper}
	    \ingredient{¼ teaspoon ground ginger}
	    \ingredient{⅛ teaspoon salt}
	    \ingredient{2 pounds skinless, boneless chicken thighs}
	    \ingredient{Cooking spray}
		
	    \ingredientsection{Sauce}
	    \ingredient{2 teaspoons canola oil}
	    \ingredient{½ cup finely chopped onion}
	    \ingredient{1 tablespoon dark brown sugar}
	    \ingredient{1 teaspoon chili powder}
	    \ingredient{½ teaspoon garlic powder}
	    \ingredient{½ teaspoon dry mustard}
	    \ingredient{¼ teaspoon ground allspice}
	    \ingredient{⅛ teaspoon ground red pepper}
	    \ingredient{1 cup ketchup}
	    \ingredient{2 tablespoons cider vinegar}
		
	    \ingredientsection{Remaining ingredients}
	    \ingredient{8 (1 1/2-ounce) hamburger buns, toasted}
	    \ingredient{16 hamburger dill chips}

	},
	instructions = {
	    \instruction{Preheat grill to medium-high heat.}
	    \instruction{To prepare chicken, combine first 6 ingredients; rub evenly over chicken. Place chicken on a grill rack coated with cooking spray; cover and grill 15 minutes or until a thermometer registers 180°, turning occasionally. Let stand for 5 minutes. Shred with 2 forks.}
	    \instruction{To prepare sauce, while chicken grills, heat canola oil in a medium saucepan over medium heat. Add onion; cook for 5 minutes or until tender, stirring occasionally. Stir in 1 tablespoon sugar and next 5 ingredients (through ground red pepper); cook 30 seconds. Stir in ketchup and vinegar; bring to a boil. Reduce heat, and simmer 10 minutes or until slightly thickened, stirring occasionally. Stir in chicken; cook 2 minutes.}
	    \instruction{Place 1/3 cup chicken mixture on bottom half of each bun; top each with 2 pickle chips and top of bun.}
	}
}

\recipe{%
    % layout={simple},
    image={../images/recipes/beef_daube_provencal.jpg},
    columnratio={0.35, 0.65},
	title = {Beef Daube Procençal},
	indexes = {Beef, Stew, Recipes!Entree, French Cuisine},
	description = {Best Beef Recipe. This classic French braised beef, red wine, and vegetable stew is simple and delicious. It stands above all of our other beef recipes because it offers the homey comfort and convenience of pot roast yet is versatile and sophisticated enough for entertaining. Garnish with chopped fresh thyme.},
	serves = {12},
	preptime = {15 mins},
	cookingtime = {10 mins},
	difficulty = {Beginner},
	origin = {USA},
	tags = {Breakfast, Pizza, Quick, Baked},
	vegetarian = {yes},
	spicy = {yes},
	extrainstructioninfo = {To make in a slow cooker, prepare through Step Place beef mixture in an electric slow cooker. Cover and cook on HIGH for 5 hours.},
	ingredients = {
	    \ingredient{2 teaspoons olive oil}
	    \ingredient{12 garlic cloves, crushed}
	    \ingredient{1 (2-pound) boneless chuck roast, trimmed and cut into 2-inch cubes}
	    \ingredient{1 1/2 teaspoons salt, divided}
	    \ingredient{1/2 teaspoon freshly ground black pepper, divided}
	    \ingredient{1 cup red wine}
	    \ingredient{2 cups chopped carrot}
	    \ingredient{1 1/2 cups chopped onion}
	    \ingredient{1/2 cup less-sodium beef broth}
	    \ingredient{1 tablespoon tomato paste}
	    \ingredient{1 teaspoon chopped fresh rosemary}
	    \ingredient{1 teaspoon chopped fresh thyme}
	    \ingredient{Dash of ground cloves}
	    \ingredient{1 (14.5-ounce) can diced tomatoes, undrained}
	    \ingredient{1 bay leaf}
	    \ingredient{3 cups hot cooked medium egg noodles (about 4 cups uncooked noodles)}
	    \ingredient{Chopped fresh thyme (optional)}
	},
	instructions = {
	    \instruction{Preheat oven to 300°.}
		\instruction{Heat olive oil in a small Dutch oven over low heat. Add garlic to pan; cook for 5 minutes or until garlic is fragrant, stirring occasionally. Remove garlic with a slotted spoon; set aside. Increase heat to medium-high. Add beef to pan. Sprinkle beef with 1/2 teaspoon salt and 1/4 teaspoon black pepper. Cook 5 minutes, browning on all sides. Remove beef from pan. Add wine to pan, and bring to a boil, scraping pan to loosen browned bits. Add garlic, beef, remaining 1 teaspoon salt, remaining 1/4 teaspoon pepper, carrot, and next 8 ingredients (through bay leaf) to pan; bring to a boil.}
		\instruction{Cover and bake at 300° for 2 1/2 hours or until beef is tender. Discard bay leaf. Serve over noodles. Garnish with chopped fresh thyme, if desired.}
	}
}

\recipe{%
    % layout={simple},
    image={../images/recipes/beef_rendang.jpg},
    columnratio={0.35, 0.65},
	title = {Beef Rendang},
	indexes = {Beef, Stew, Recipes!Entree, Malasian Cuisine},
	description = {This rich Malay curry features aromatic lemongrass, garlic, ginger, and cinnamon. Make sure the beef mixture cooks at a low simmer so the sauce doesn't scorch and the meat slowly becomes tender. If you can't find unsweetened coconut, use sweetened flaked coconut and omit the added sugar.},
	serves = {6},
	preptime = {15 mins},
	cookingtime = {10 mins},
	difficulty = {Beginner},
	origin = {USA},
	tags = {Breakfast, Pizza, Quick, Baked},
	% vegetarian = {yes},
	spicy = {yes},
	extrainstructioninfo = {Place additional comments here.},
	ingredients = {
	    \ingredient{1/2 cup chopped shallots}
        \ingredient{1/3 cup thinly sliced peeled ginger}
        \ingredient{1 1/2 tablespoons minced garlic (about 5 cloves)}
        \ingredient{2 tablespoons chili garlic sauce (such as Lee Kum Kee)}
        \ingredient{1 1/2 teaspoons ground turmeric}
        \ingredient{1 1/4 teaspoons salt}
        \ingredient{1/4 teaspoon ground cinnamon}
        \ingredient{6 whole cloves}
        \ingredient{1 to 2 serrano chiles, chopped}
        \ingredient{1 (14-ounce) can light coconut milk, divided}
        \ingredient{2/3 cup flaked unsweetened coconut, toasted}
        \ingredient{1 teaspoon grated lime rind}
        \ingredient{2 tablespoons fresh lime juice}
        \ingredient{2 teaspoons sugar}
        \ingredient{2 (3-inch) fresh lemongrass stalks, crushed}
        \ingredient{2 pounds boneless chuck roast, trimmed and cut into 1-inch cubes}
        \ingredient{1 (14-ounce) can fat-free, less-sodium chicken broth}
        \ingredient{4 cups hot cooked basmati rice}

	},
	instructions = {
	    \instruction{Place first 9 ingredients in a food processor or mini chopper. Add 1/4 cup coconut milk; process until smooth. Spoon mixture into a bowl; set aside.}
		\instruction{Place 3 tablespoons coconut milk and flaked coconut in food processor; process until a smooth paste forms.}
		\instruction{Heat a large saucepan over medium-high heat. Add shallot mixture; cook 1 minute or until fragrant, stirring constantly. Stir in remaining coconut milk, rind, and next 5 ingredients (through broth); bring to a boil. Cover, reduce heat to medium-low, and simmer 1 1/2 hours or until beef is tender, stirring occasionally. Discard lemongrass. Stir in flaked coconut mixture; simmer 10 minutes or until liquid almost evaporates. Serve over rice.}
	}
}

\recipe{%
    % layout={simple},
    % image={../images/recipes/gravy_pizza.png},
    columnratio={0.35, 0.65},
	title = {Slow Simmered Meat Sauce},
	indexes = {Pasta, Sauce, Recipes!Entree, Italian Cuisine},
	description = {Mafaldine is a flat noodle with ruffled edges. You can substitute spaghetti.},
	serves = {8},
	preptime = {30 mins},
	cookingtime = {8 hr},
	difficulty = {Beginner},
	origin = {USA},
	tags = {Sauce, Pastz, Slow-Cooker},
	% extrainstructioninfo = {Place additional comments here.},
	ingredients = {
	    \ingredient{1 tablespoon olive oil}
	    \ingredient{2 cups chopped onion}
	    \ingredient{1 cup chopped carrot}
	    \ingredient{6 garlic cloves, minced}
	    \ingredient{2 (4-ounce) links hot Italian sausage, casings removed}
	    \ingredient{1 pound ground sirloin}
	    \ingredient{½ cup kalamata olives, pitted and sliced}
	    \ingredient{¼ cup no-salt-added tomato paste}
	    \ingredient{1 ½ teaspoons sugar}
	    \ingredient{1 teaspoon kosher salt}
	    \ingredient{½ teaspoon crushed red pepper}
	    \ingredient{1 (28-ounce) can no-salt-added crushed tomatoes, undrained}
	    \ingredient{1 cup no-salt-added tomato sauce}
	    \ingredient{1 tablespoon chopped fresh oregano}
	    \ingredient{16 ounces uncooked mafaldine pasta}
	    \ingredient{½ cup torn fresh basil}
	    \ingredient{3 ounces shaved fresh Parmigiano-Reggiano cheese}
	},
	instructions = {
	    \instruction{Heat a large skillet over medium-high heat. Add oil to pan; swirl to coat. Add onion and carrot to pan; sauté 4 minutes, stirring occasionally. Add garlic; sauté 1 minute, stirring constantly. Place vegetable mixture in a 6-quart slow cooker. Add sausage and beef to skillet; sauté 6 minutes or until browned, stirring to crumble. Remove beef mixture from skillet using a slotted spoon. Place beef mixture on a double layer of paper towels; drain. Add beef mixture to slow cooker. Stir olives and next 6 ingredients (through tomato sauce) into slow cooker. Cover and cook on LOW 8 hours. Stir in oregano.}
		\instruction{Prepare pasta according to package directions, omitting salt and fat. Serve sauce with hot cooked pasta; top with basil and cheese.}
	}
}

\recipe{%
    % layout={simple},
    % image={../images/recipes/foo.png},
    columnratio={0.35, 0.65},
	title = {Gnocchi Bolognese},
	indexes = {Beef, Stew, Recipes!Entree, Italian Cuisine},
	description = {Recipe Description here.},
	serves = {12},
	preptime = {15 mins},
	cookingtime = {10 mins},
	difficulty = {Beginner},
	origin = {USA},
	tags = {Breakfast, Pizza, Quick, Baked},
	% vegetarian = {yes},
	spicy = {yes},
	extrainstructioninfo = {Place additional comments here.},
	ingredients = {
	    \ingredientsection{Section 1}
	    \ingredient{First Ingredient}
	},
	instructions = {
	    \instructionsection{First Section}
	    \instruction{Place the text for the first step here}
	}
}


%----------------------------------------------------------------------------------------
%	DESSERTS CHAPTER PAGE
%----------------------------------------------------------------------------------------

\makechapterpage{
	title={Desserts},
	image={../images/book/dessert.jpg},
	layout={right}
}

\recipe{%
    % layout={simple},
    % image={../images/recipes/foo.png},
    columnratio={0.35, 0.65},
	title = {Chocolate Cake with Mascarpone Frosting},
	% indexes = {Pizza, Eggs, Recipes!Breakfast},
	description = {Recipe Description here.},
	serves = {12},
	preptime = {15 mins},
	cookingtime = {10 mins},
	difficulty = {Beginner},
	origin = {USA},
	tags = {Breakfast, Pizza, Quick, Baked},
	vegetarian = {yes},
	% spicy = {yes},
	extrainstructioninfo = {Place additional comments here.},
	ingredients = {
	    \ingredientsection{Section 1}
	    \ingredient{First Ingredient}
	},
	instructions = {
	    \instructionsection{First Section}
	    \instruction{Place the text for the first step here}
	}
}

\recipe{%
    % layout={simple},
    % image={../images/recipes/foo.png},
    columnratio={0.35, 0.65},
	title = {Strawberry Cake},
	indexes = {Pizza, Eggs, Recipes!Breakfast},
	description = {Recipe Description here.},
	serves = {12},
	preptime = {15 mins},
	cookingtime = {10 mins},
	difficulty = {Beginner},
	origin = {USA},
	tags = {Breakfast, Pizza, Quick, Baked},
	% vegetarian = {yes},
	% spicy = {yes},
	extrainstructioninfo = {Place additional comments here.},
	ingredients = {
	    \ingredientsection{Section 1}
	    \ingredient{First Ingredient}
	},
	instructions = {
	    \instructionsection{First Section}
	    \instruction{Place the text for the first step here}
	}
}

%----------------------------------------------------------------------------------------
%	CONVERSION TABLE
%----------------------------------------------------------------------------------------

\makeconversionpage{
	title={Conversion Tables}
}

%----------------------------------------------------------------------------------------
%	INDEX
%----------------------------------------------------------------------------------------

\printindex

%----------------------------------------------------------------------------------------
%	BACK COVER
%----------------------------------------------------------------------------------------

\makebackcoverpage{
	topcontent={
		{\fontsize{24pt}{28pt}\sourcesanspro\bfseries\selectfont\color{white}\MakeUppercase{About This Book}}\par
		\vspace{0.02\textheight}
		This cookbook represents a collection of cherished recipes passed down through generations, each one telling a story of family gatherings, holiday celebrations, and everyday moments made special by the food we share. From simple comfort foods to elaborate feasts, these recipes have been tested, refined, and perfected over countless meals.\newline\newline
		Whether you're a seasoned cook or just beginning your culinary journey, we hope these recipes bring joy, inspiration, and delicious results to your kitchen.
	},
	image={../images/book/back-cover.jpg},
	imageopacity={0.8},
	imageposition={right},
	columnratio={0.5,0.5},
	verticalsplit={0.5},
	bottomcontent={
		\textbf{From Our Kitchen to Yours:}\par
		\vspace{0.01\textheight}
		Start your morning right with our fluffy \textbf{Banana Pancakes}—a family favorite that's both simple and satisfying. For a special weekend treat, you simply must try our \textbf{Classic French Toast}, golden and perfectly crisp.\newline\newline
		When it comes to main courses, our \textbf{Spaghetti Bolognese} has been a Sunday dinner tradition for decades, while the \textbf{Lemon Herb Grilled Salmon} brings elegance to any weeknight meal. And don't miss our family's favorite dessert—the \textbf{Classic Tiramisu} that has graced countless celebrations and always leaves guests asking for the recipe.\newline\newline
		\textit{Each recipe includes detailed instructions, ingredient lists, and helpful tips to ensure your success in the kitchen.}
	},
	isbn={978-0-123456-78-9},
	publisher={Published by Your Publisher Name},
	copyright={© 2025 All rights reserved.},
	textcolor={white},
	bgcolor={darkgrey},
	divider={true},
	barcodeplaceholder={true}
}

\end{document}
