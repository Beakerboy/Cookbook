\documentclass[
	letterpaper,
	10pt,
	twoside
]{CookBook}

\begin{document}

\makecoverpage{
	title={My Cookbook},
	subtitle={Our Favorite Recipes},
	author={Kevin Nowaczyk},
	titlefontsize={\fontsize{36pt}{38pt}},
	subtitlefontsize={\fontsize{24pt}{26pt}},
	image={../images/book/cover.jpg},
	opacity={0.6},
	bgcolor={darkgrey},
	textcolor={white},
	shadowoffset={0.05cm}
}

%----------------------------------------------------------------------------------------
%	PREFACE
%----------------------------------------------------------------------------------------

\makeprefacepage{
	title={Preface},
	text={Welcome to this collection of recipes, gathered from three generations of family cooking and culinary adventures around the world. Each recipe tells a story—some passed down through handwritten notes on yellowed paper, others discovered during travels through bustling markets and quiet countryside kitchens.\newline

	This cookbook is more than just a collection of instructions and ingredients. It's a celebration of the joy that comes from creating something delicious, the warmth of sharing a meal with loved ones, and the memories that form around the dinner table. From simple weekday breakfasts to elaborate weekend feasts, these recipes have been tested, adjusted, and perfected over countless meals.\newline

	Whether you're a beginner just learning your way around the kitchen or an experienced cook looking for new inspiration, I hope these recipes bring as much happiness to your table as they have brought to ours. Don't be afraid to make them your own—the best recipes are the ones that evolve with each telling.\newline

	Happy cooking!},
	layout={single},
	image={../images/book/preface.jpg}
}

%----------------------------------------------------------------------------------------
%	TABLE OF CONTENTS
%----------------------------------------------------------------------------------------

\maketoc

%----------------------------------------------------------------------------------------
%	BREAKFAST CHAPTER PAGE
%----------------------------------------------------------------------------------------

\makechapterpage{
	title={Breakfast},
	bgcolor={paleorange},
	image={../images/book/breakfast.jpg},
	layout={right}
}


%----------------------------------------------------------------------------------------------------------


%----------------------------------------------------------------------------------------
%	RECIPE EXAMPLE - Breakfast Sausage Casserole
%----------------------------------------------------------------------------------------

\recipe{%
	layout={simple},
	imageheight={0.25\paperheight},
	imageoverlayspace={0.2\paperheight},
	title = {Breakfast Sausage Casserole},
	description = {This satisfying recipe is perfect to make for weekend guests. Assemble and refrigerate the casserole the night before, and just pop it in the oven the next morning. Look for turkey sausage near other breakfast-style sausage in the frozen foods section.},
	serves = {8},
	preptime = {15 mins},
	cookingtime = {65 mins},
	difficulty = {Beginner},
	origin = {USA},
	tags = {Breakfast},
	indexes = {Breakfast Casserole, Recipes!Breakfast, American cuisine},
	ingredients = {
	    \ingredient{1 (16-ounce) package sage sausage}
		},
	instructions = {
	    \instruction{Heat a large nonstick skillet over medium-high heat. Coat pan with cooking spray. Add sausage to pan, cook 5 minutes or until browned, stirring well to crumble.}
	}
}


%----------------------------------------------------------------------------------------------------------


%----------------------------------------------------------------------------------------
%	RECIPE EXAMPLE - Bacon, Ham, and Egg Hash (Columns Layout, No Image)
%----------------------------------------------------------------------------------------

\recipe{%
    image = {../images/recipes/hash.jpeg},
	title = {Bacon, Ham, and Egg Hash},
	indexes = {Hash, Recipes!Breakfast, Potato, Eggs, Quick meals},
	description = {Golden, crispy French toast with a custardy center. A weekend breakfast favorite that's easy to make yet feels special. Perfect with maple syrup and fresh berries.},
	serves = {4},
	preptime = {10 mins},
	cookingtime = {15 mins},
	difficulty = {Beginner},
	origin = {USA},
	tags = {Breakfast, Quick},
	ingredients = {
			\ingredientsection{Custard Mixture}
			\ingredient{4 large eggs}
			\ingredient{1 cup whole milk}
			\ingredient{2 tbsp granulated sugar}
			\ingredient{1 tsp vanilla extract}
			\ingredient{1 tsp ground cinnamon}
			\ingredient{1/4 tsp ground nutmeg}
			\ingredient{Pinch of salt}

			\ingredientsection{Cooking and Serving}
			\ingredient{8 thick slices bread (brioche, challah, or white bread)\note{Day-old bread works best as it absorbs the custard mixture without falling apart. Brioche or challah bread creates the most luxurious French toast, but any thick-sliced bread will work.}}
			\ingredient{2-3 tbsp butter for cooking}
			\ingredient{Maple syrup, for serving}
			\ingredient{Fresh berries, for serving}
			\ingredient{Powdered sugar for dusting (optional)}
		},
	instructions ={
			\instruction{In a shallow bowl or baking dish, whisk together eggs, milk, sugar, vanilla extract, cinnamon, nutmeg, and salt until well combined.}
			\instruction{Heat a large skillet or griddle over medium heat and add 1 tablespoon of butter.}
			\instruction{Dip each bread slice into the custard mixture, allowing it to soak for about 5 seconds per side. Don't oversoak or the bread will fall apart.}
			\instruction{Place soaked bread slices in the heated skillet. Cook for 2-3 minutes per side until golden brown and crispy on the outside.}
			\instruction{Transfer cooked French toast to a plate and keep warm. Repeat with remaining bread slices, adding more butter to the pan as needed.}
			\instruction{Serve immediately topped with maple syrup, fresh berries, and a dusting of powdered sugar if desired.}
		}
}

%----------------------------------------------------------------------------------------------------------


%----------------------------------------------------------------------------------------
%	APPETIZERS CHAPTER PAGE
%----------------------------------------------------------------------------------------

\makechapterpage{
	title={Appetizers},
	image={../images/book/salad.jpg}
}

%----------------------------------------------------------------------------------------------------------


%----------------------------------------------------------------------------------------
%	RECIPE - Esquites (Columns Layout)
%----------------------------------------------------------------------------------------

\recipe{%
    image = {../images/recipes/esquites.jpeg},
	title = {Mexican Street Corn (Esquites)},
	indexes = {Recipes!Appetizers, Corn, Vegetarian recipes, Mexican cuisine},
	description = {From SeriousEats},
	serves = {4},
	preptime = {5 mins},
	cookingtime = {15 mins},
	difficulty = {Beginner},
	origin = {Mexico},
	tags = {Mexican, Quick, Vegetarian},
	vegetarian = {yes},
	ingredients = {
			\ingredient{2 tablespoon vegtable oil}
			\ingredient{4 ears fresh corn (about 3 cups) or 2 cans}
			\ingredient{kosher salt}
			\ingredient{2 ounces fet or corija cheese, finely crumbled}
			\ingredient{1/2 cup finely sliced scallions}
			\ingredient{1/2 cup cilantro, finely chopped}
			\ingredient{1 jalapeño pepper, finely chopped }
			\ingredient{2 tablespoons mayonnaise}
			\ingredient{1 tablespoon fresh lime juice}
			\ingredient{chili powder to taste}
		},
	instructions ={
			\instruction{Heat oil in a large nonstick skillet or wok over high heat until shimmering. Add corn kernels, season to taste with salt, toss once or twice, and cook without moving until charred on one side, about 2 minutes. Toss corn, stir, and repeat until charred on second side, about 2 minutes longer. Continue tossing and charring until corn is well charred all over, about 10 minutes total. Transfer to a large bowl.}
			\instruction{Add cheese, scallions, cilantro, jalapeño, garlic, mayonnaise, lime juice, and chile powder and toss to combine. Taste and adjust seasoning with salt and more chile powder to taste. Serve immediately.}
		}
}

\recipe{%
    image = {../images/recipes/BG_pasta_salad.jpg},
	title = {Broccoli, Grape, and Pasta Salad},
	description = {If you're a broccoli salad fan, you'll love the combination of these colorful ingredients. Cook the pasta al dente so it's firm enough to hold its own when tossed with the tangy-sweet salad dressing.},
	ingredients = {
	    \ingredient{1 cup chopped pecans}
	},
	instructions = {
	    \instruction{Preheat oven to 350°. Bake pecans in a single layer in a shallow pan 5 to 7 minutes or until lightly toasted and fragrant, stirring halfway through.}
	}
}

\recipe{%
    image = {../images/recipes/potato_salad.jpg},
	title = {Roasted Potato Salad with Mustard Dressing},
	description = {This tangy side dish with sweet onions and honey pairs beautifully with burgers or steak. This salad is best chilled but can stay at room temperature for up to two hours. You can also use sweet-hot mustard in place of the Dijon and honey for a zesty flavor.},
	serves = {8},
	ingredients = {
	    \ingredient{3 pounds small red potatoes, cut into 1-inch pieces}
		\ingredient{}
	    \ingredient{}
	    \ingredient{}
	    \ingredient{}
	    \ingredient{}
	    \ingredient{}
	    \ingredient{}
	    \ingredient{}
	    \ingredient{}
	    \ingredient{}
	    \ingredient{}
	},
	instructions = {
	    \instruction{Preheat oven to 400°}
	}
}

\recipe{%
	title = {Sourdough Stuffing with Pears and Sausage},
	description = {Sourdough bread gives the stuffing a tangier flavor than French bread, but you can use the latter in a pinch.},
	serves = {12},
	ingredients = {
	    \ingredient{8 cups (½-inch) cubed sourdough bread (about 12 ounces)}
	},
	instructions = {
	    \instruction{Preheat oven to 425°}
	}
}

%----------------------------------------------------------------------------------------
%	ENTREES CHAPTER PAGE
%----------------------------------------------------------------------------------------

\makechapterpage{
	title={Entrees},
	image={../images/book/pasta.jpg}
}

%----------------------------------------------------------------------------------------------------------


%----------------------------------------------------------------------------------------
%	RECIPE - Fajitas
%----------------------------------------------------------------------------------------

\recipe{%
	title = {Beef and Chicken Fajitas with Peppers and Onions},
	indexes = {Spaghetti Bolognese, Bolognese, Recipes!Entrees, Recipes!Pasta, Pasta, Beef, Tomatoes, Italian cuisine, Make-ahead meals},
	description = {Bolognese sauce, known in Italian as ragú alla Bolognese, is a meat-based sauce originating from Bologna, Italy. In Italian cuisine, it is customarily used to dress ``tagliatelle al ragú" and to prepare ``lasagne alla bolognese".},
	serves = {4},
	preptime = {25 mins},
	cookingtime = {40 mins},
	difficulty = {Beginner},
	origin = {Italy},
	tags = {Pasta, Meat, Italian, Bolognese, Make-Ahead},
	ingredients = {
			\ingredientsection{Marinade}
			\ingredient{1⁄4 cup olive oil}
			\ingredient{1 teaspoon lime rind,\\ grated}
			\ingredient{2 1⁄2 tablespoons lime juice}
			\ingredient{2 tablespoons Worcestershire sauce}
			\ingredient{1 1⁄2 teaspoons ground cumin}
			\ingredient{1 teaspoon salt}
			\ingredient{1⁄2 teaspoon dried oregano}
			\ingredient{1⁄2 teaspoon fresh coarse ground black pepper}
			\ingredient{3 garlic cloves,\\ minced}
			\ingredient{1 (14 1/4 ounce) can low sodium beef broth}

			\ingredientsection{Fajitas}
			\ingredient{1 (1 pound) flank steak}
			\ingredient{1 pound skinned, boned chicken breast}
			\ingredient{2 red bell peppers,\\ each cut into 12 wedges}
			\ingredient{2 green bell peppers,\\ cut into 16 wedges}
			\ingredient{1 large  Vidalia onion,\\ each cut into 12 wedges}
			\ingredient{cooking spray}
			\ingredient{16 (6 inch) flour tortillas}
			\ingredient{salsa}
			\ingredient{sour cream}
			\ingredient{cilantro}
			\ingredient{cheese}
			\ingredient{guacamole}
		},
	instructions = {
			\instructionsection{Preparing the Marinade}
			\instruction{Combine first 10 ingredients in a large bowl; set aside.}
			\instructionsection{Fajitas}
			\instruction{To prepare fajitas, trim fat from steak. Score a diamond pattern on both sides of the steak. Combine 1 1/2 cups marinade, steak, and chicken in a large zip-top plastic bag. Seal and marinate in refrigerator 4 hours or overnight, turning occasionally. Combine remaining marinade, bell peppers, and onion in a zip-top plastic bag. Seal and marinate in refrigerator for 4 hours or overnight, turning occasionally.}
			\instruction{Prepare grill.}
			\instruction{Remove steak and chicken from bag; discard marinade. Remove vegetables from bag; reserve marinade. Place reserved marinade in a small saucepan; set aside. Place steak, chicken, and vegetables on grill rack coated with cooking spray; cook 8 minutes on each side or until desired degree of doneness.}
			\instruction{Wrap tortillas tightly in foil; place tortilla packet on grill rack the last 2 minutes of grilling time. Bring reserved marinade to a boil. Cut steak and chicken diagonally across the grain into thin slices. Place the steak, chicken, and vegetables on a serving platter; drizzle with reserved marinade.}
			\instruction{Arrange about 1 ounce steak, about 1 ounce chicken, 3 bell pepper wedges, and 1 onion wedge in a tortilla; top with 1 tablespoon salsa, about 1 teaspoon sour cream, and 1/2 tablespoon cilantro.
Fold sides of tortilla over filling. Garnish with cilantro sprigs, if desired. Serve immediately.}
		}
}

%----------------------------------------------------------------------------------------
%	DESSERTS CHAPTER PAGE
%----------------------------------------------------------------------------------------

\makechapterpage{
	title={Desserts},
	image={../images/book/dessert.jpg},
	layout={right}
}

%----------------------------------------------------------------------------------------
%	CONVERSION TABLE
%----------------------------------------------------------------------------------------

\makeconversionpage{
	title={Conversion Tables}
}

%----------------------------------------------------------------------------------------
%	INDEX
%----------------------------------------------------------------------------------------

\printindex

%----------------------------------------------------------------------------------------
%	BACK COVER
%----------------------------------------------------------------------------------------

\makebackcoverpage{
	topcontent={
		{\fontsize{24pt}{28pt}\sourcesanspro\bfseries\selectfont\color{white}\MakeUppercase{About This Book}}\par
		\vspace{0.02\textheight}
		This cookbook represents a collection of cherished recipes passed down through generations, each one telling a story of family gatherings, holiday celebrations, and everyday moments made special by the food we share. From simple comfort foods to elaborate feasts, these recipes have been tested, refined, and perfected over countless meals.\newline\newline
		Whether you're a seasoned cook or just beginning your culinary journey, we hope these recipes bring joy, inspiration, and delicious results to your kitchen.
	},
	image={../images/book/back-cover.jpg},
	imageopacity={0.8},
	imageposition={right},
	columnratio={0.5,0.5},
	verticalsplit={0.5},
	bottomcontent={
		\textbf{From Our Kitchen to Yours:}\par
		\vspace{0.01\textheight}
		Start your morning right with our fluffy \textbf{Banana Pancakes}—a family favorite that's both simple and satisfying. For a special weekend treat, you simply must try our \textbf{Classic French Toast}, golden and perfectly crisp.\newline\newline
		When it comes to main courses, our \textbf{Spaghetti Bolognese} has been a Sunday dinner tradition for decades, while the \textbf{Lemon Herb Grilled Salmon} brings elegance to any weeknight meal. And don't miss our family's favorite dessert—the \textbf{Classic Tiramisu} that has graced countless celebrations and always leaves guests asking for the recipe.\newline\newline
		\textit{Each recipe includes detailed instructions, ingredient lists, and helpful tips to ensure your success in the kitchen.}
	},
	isbn={978-0-123456-78-9},
	publisher={Published by Your Publisher Name},
	copyright={© 2025 All rights reserved.},
	textcolor={white},
	bgcolor={darkgrey},
	divider={true},
	barcodeplaceholder={true}
}

\end{document}
