\documentclass[
	letterpaper,
	10pt,
	twoside
]{CookBook}
\usepackage{nicefrac}
\edef\tsp{t }
\edef\tbl{T }
\edef\cu{c }

\begin{document}

\makecoverpage{
	title={My Cookbook},
	subtitle={Our Favorite Recipes},
	author={Kevin Nowaczyk},
	titlefontsize={\fontsize{36pt}{38pt}},
	subtitlefontsize={\fontsize{24pt}{26pt}},
	image={../images/book/cover.jpg},
	opacity={0.6},
	bgcolor={darkgrey},
	textcolor={white},
	shadowoffset={0.05cm}
}

%----------------------------------------------------------------------------------------
%	PREFACE
%----------------------------------------------------------------------------------------

\makeprefacepage{
	title={Preface},
	text={Welcome to this collection of recipes, gathered from three generations of family cooking and culinary adventures around the world. Each recipe tells a story—some passed down through handwritten notes on yellowed paper, others discovered during travels through bustling markets and quiet countryside kitchens.\newline

	This cookbook is more than just a collection of instructions and ingredients. It's a celebration of the joy that comes from creating something delicious, the warmth of sharing a meal with loved ones, and the memories that form around the dinner table. From simple weekday breakfasts to elaborate weekend feasts, these recipes have been tested, adjusted, and perfected over countless meals.\newline

	Whether you're a beginner just learning your way around the kitchen or an experienced cook looking for new inspiration, I hope these recipes bring as much happiness to your table as they have brought to ours. Don't be afraid to make them your own—the best recipes are the ones that evolve with each telling.\newline

	Happy cooking!},
	layout={single},
	image={../images/book/preface.jpg}
}

%----------------------------------------------------------------------------------------
%	TABLE OF CONTENTS
%----------------------------------------------------------------------------------------

\maketoc

%----------------------------------------------------------------------------------------
%	BREAKFAST CHAPTER PAGE
%----------------------------------------------------------------------------------------

\makechapterpage{
	title={Breakfast},
	bgcolor={paleorange},
	image={../images/book/breakfast.jpg},
	layout={right}
}


%----------------------------------------------------------------------------------------------------------


%----------------------------------------------------------------------------------------
%	RECIPE EXAMPLE - Breakfast Sausage Casserole
%----------------------------------------------------------------------------------------

\recipe{%
	% layout={simple},
	image = {../images/recipes/sausage_casserole.jpeg},
	imageheight={0.25\paperheight},
	imageoverlayspace={0.2\paperheight},
	title = {Breakfast Sausage Casserole},
	description = {This satisfying recipe is perfect to make for weekend guests. Assemble and refrigerate the casserole the night before, and just pop it in the oven the next morning. Look for turkey sausage near other breakfast-style sausage in the frozen foods section. From Cooking Light.},
	serves = {8},
	preptime = {15 mins},
	cookingtime = {65 mins},
	difficulty = {Beginner},
	origin = {USA},
	tags = {Breakfast},
	indexes = {Breakfast Casserole, Recipes!Breakfast, American cuisine},
	ingredients = {
	    \ingredient{1 (16-ounce) package sage sausage}
		\ingredient{8 (1½-ounce) slices sourdough bread, cut into ½-inch cubes (about 8 cups)}
		\ingredient{⅔ \cu (about 2½ ounces) shredded sharp cheddar cheese}
		\ingredient{3 \cu 1\% low-fat milk, divided}
		\ingredient{1 \cu egg substitute}
		\ingredient{1 \tbl Dijon mustard}
		\ingredient{1 (10.75-ounce) can condensed 30\% reduced-sodium, 98\% fat-free cream of mushroom soup, undiluted}
		},
	instructions = {
	    \instruction{Heat a large nonstick skillet over medium-high heat. Coat pan with cooking spray. Add sausage to pan, cook 5 minutes or until browned, stirring well to crumble.}
		\instruction{Arrange bread in a 13 x 9-inch baking dish coated with cooking spray. Top evenly with cooked turkey sausage and cheddar cheese. Combine 2½ cups milk, egg substitute, and Dijon mustard, stirring with a whisk. Pour over bread mixture in dish. Cover and refrigerate 8 hours or overnight.}
		\instruction{Preheat oven to 350°.}
		\instruction{Uncover casserole. Combine remaining ½ cup milk and cream of mushroom soup, stirring with a whisk. Pour over bread mixture. Bake at 350° for 1 hour and 5 minutes or until set and lightly browned. Let stand 15 minutes before serving.}
	}
}


%----------------------------------------------------------------------------------------------------------


%----------------------------------------------------------------------------------------
%	RECIPE EXAMPLE - Bacon, Ham, and Egg Hash (Columns Layout, No Image)
%----------------------------------------------------------------------------------------

\recipe{%
    image = {../images/recipes/hash.jpeg},
	title = {Bacon, Ham, and Egg Hash},
	indexes = {Hash, Recipes!Breakfast, Potato, Eggs, Quick meals},
	description = {The bacon seasons the potatoes in this hearty skillet meal. Make it for a weekend breakfast, and serve with fresh fruit.},
	serves = {4},
	preptime = {5 mins},
	cookingtime = {30 mins},
	difficulty = {Beginner},
	origin = {USA},
	tags = {Breakfast, Quick},
	ingredients = {
			\ingredient{4 slices bacon}
			\ingredient{2 large potatoes, peeled and chopped (2 ½ \cu)}
			\ingredient{cooking oil}
			\ingredient{8 ounces cooked ham, coarsely chopped}
			\ingredient{5 eggs, lightly beaten}
			\ingredient{¼ \tsp salt}
			\ingredient{\nicefrac{1}{8} \tsp ground black pepper}
			\ingredient{¼ cup shredded cheddar cheese (1 oz.)}
		},
	instructions ={
			\instruction{In a large skillet, cook bacon until crisp, reserving fat. Remove bacon from skillet; drain well on paper towels and crumble. Set aside.}
			\instruction{Add potatoes to the reserved fat and add additional oil if there's not enough fat. Cook, uncovered, over medium heat about 10 minutes or until potatoes are almost tender and browned, turning frequently. Add the ham. Cook about 5 minutes more or until potatoes are tender, stirring gently.}
			\instruction{In a small bowl, combine eggs, salt and black pepper. Pour egg mixture over potato-ham mixture. Cook over medium heat, without stirring, until egg mixture begins to set on the bottom and around edge. Using a spatula or a large spoon, lift and fold the partially cooked egg mixture so that the uncooked portion flows underneath.}
			\instruction{Continue cooking over medium heat until egg mixture is cooked through but is still glossy and moist. Sprinkle with cheese and reserved bacon. Remove from heat. Cover and let stand for 1 to 2 minutes or until cheese is melted.}
		}
}

\recipe{%
    %layout={simple},
    image={../images/recipes/gravy_pizza.png},
    columnratio={0.35, 0.65},
	title = {Gravy Pizza},
	indexes = {Pizza, Eggs, Recipes!Breakfast},
	description = {Recipe Description here.},
	serves = {12},
	preptime = {15 mins},
	cookingtime = {10 mins},
	difficulty = {Beginner},
	origin = {USA},
	tags = {Breakfast, Pizza, Quick, Baked},
	% vegetarian = {yes},
	% spicy = {yes},
	extrainstructioninfo = {Place additional comments here.},
	ingredients = {
	    \ingredientsection{Section 1}
	    \ingredient{First Ingredient}
	},
	instructions = {
	    \instructionsection{First Section}
	    \instruction{Place the text for the first step here}
	}
}

%----------------------------------------------------------------------------------------------------------


%----------------------------------------------------------------------------------------
%	APPETIZERS CHAPTER PAGE
%----------------------------------------------------------------------------------------

\makechapterpage{
	title={Appetizers},
	image={../images/book/salad.jpg}
}

%----------------------------------------------------------------------------------------------------------


%----------------------------------------------------------------------------------------
%	RECIPE - Esquites (Columns Layout)
%----------------------------------------------------------------------------------------

\recipe{%
    image = {../images/recipes/esquites.jpeg},
	title = {Mexican Street Corn (Esquites)},
	indexes = {Recipes!Appetizers, Corn, Vegetarian recipes, Mexican cuisine},
	description = {From SeriousEats},
	serves = {4},
	preptime = {5 mins},
	cookingtime = {15 mins},
	difficulty = {Beginner},
	origin = {Mexico},
	tags = {Mexican, Quick, Vegetarian},
	vegetarian = {yes},
	ingredients = {
			\ingredient{2 \tbl vegtable oil}
			\ingredient{4 ears fresh corn (about 3 \cu) or 2 cans}
			\ingredient{kosher salt}
			\ingredient{2 ounces feta or corija cheese, finely crumbled}
			\ingredient{½ \cu finely sliced scallions}
			\ingredient{½ \cu cilantro, finely chopped}
			\ingredient{1 jalapeño pepper, finely chopped }
			\ingredient{2 \tbl mayonnaise}
			\ingredient{1 \tbl fresh lime juice}
			\ingredient{chili powder to taste}
		},
	instructions ={
			\instruction{Heat oil in a large nonstick skillet or wok over high heat until shimmering. Add corn kernels, season to taste with salt, toss once or twice, and cook without moving until charred on one side, about 2 minutes. Toss corn, stir, and repeat until charred on second side, about 2 minutes longer. Continue tossing and charring until corn is well charred all over, about 10 minutes total. Transfer to a large bowl.}
			\instruction{Add cheese, scallions, cilantro, jalapeño, garlic, mayonnaise, lime juice, and chile powder and toss to combine. Taste and adjust seasoning with salt and more chile powder to taste. Serve immediately.}
		}
}

\recipe{%
    image = {../images/recipes/BG_pasta_salad.jpg},
	title = {Broccoli, Grape, and Pasta Salad},
	description = {If you're a broccoli salad fan, you'll love the combination of these colorful ingredients. Cook the pasta al dente so it's firm enough to hold its own when tossed with the tangy-sweet salad dressing. From Southern Living.},
	serves = {6-8},
	preptime = {25 mins},
	cookingtime = {3 hours 30 mins},
	ingredients = {
	    \ingredient{1 \cu chopped pecans}
		\ingredient{½ (16-oz.) package farfalle (bow-tie) pasta}
		\ingredient{1 lb. fresh broccoli}
		\ingredient{1 \cu mayonnaise}
		\ingredient{\nicefrac{1}{3} \cu sugar}
		\ingredient{\nicefrac{1}{3} \cu diced red onion}
		\ingredient{\nicefrac{1}{3} \cu red wine vinegar}
		\ingredient{1 \tsp salt}
		\ingredient{2 \cu seedless red grapes, halved}
		\ingredient{8 cooked bacon slices, crumbled}
	},
	instructions = {
	    \instruction{Preheat oven to 350°. Bake pecans in a single layer in a shallow pan 5 to 7 minutes or until lightly toasted and fragrant, stirring halfway through.}
		\instruction{Prepare pasta according to package directions.}
		\instruction{Meanwhile, cut broccoli florets from stems, and separate florets into small pieces using tip of a paring knife. Peel away tough outer layer of stems, and finely chop stems.}
		\instruction{Whisk together mayonnaise and next 4 ingredients in a large bowl; add broccoli, hot cooked pasta, and grapes, and stir to coat. Cover and chill 3 hours. Stir bacon and pecans into salad just before serving.}
	}
}

\recipe{%
    image = {../images/recipes/potato_salad.jpg},
	title = {Roasted Potato Salad with Mustard Dressing},
	description = {This tangy side dish with sweet onions and honey pairs beautifully with burgers or steak. This salad is best chilled but can stay at room temperature for up to two hours. You can also use sweet-hot mustard in place of the Dijon and honey for a zesty flavor. From Cooking Light},
	serves = {8},
	ingredients = {
	    \ingredient{3 pounds small red potatoes, cut into 1-inch pieces}
		\ingredient{1 \tbl olive oil}
        \ingredient{2 \tsp freshly ground black pepper}
        \ingredient{½ \tsp kosher salt}
        \ingredient{2 bacon slices (uncooked), chopped}
        \ingredient{2 \cu diced Vidalia or other sweet onion (about 2 medium)}
        \ingredient{2 garlic cloves, minced}
        \ingredient{3 \tbl Dijon mustard}
        \ingredient{2 \tbl reduced-fat mayonnaise}
        \ingredient{1 ½ \tbl honey}
        \ingredient{1 ½ \tbl sherry vinegar}
        \ingredient{¼ \cu chopped fresh parsley}

	},
	instructions = {
	    \instruction{Preheat oven to 400°}
		\instruction{Combine first 4 ingredients in a large bowl; toss to coat. Arrange potatoes in a single layer on a jelly-roll pan. Bake at 400° for 40 minutes or until potatoes are tender, stirring once. Transfer potatoes to a large bowl.}
		\instruction{Cook bacon in a large nonstick skillet over medium heat until crisp. Remove bacon from pan, reserving 1 tablespoon drippings in pan; set bacon aside. Add onion to drippings in pan; cook 15 minutes or until golden brown and caramelized, stirring frequently. Add garlic to pan; cook 30 seconds. Add onion mixture and bacon to potatoes; toss gently. Let stand 15 minutes.}
		\instruction{Combine mustard, mayonnaise, honey, and vinegar in a small bowl; stir with a whisk. Add mustard mixture and parsley to potato mixture; toss gently.}
	}
}

\recipe{%
    image = {../images/recipes/sourdough_stuffing.jpg},
	title = {Sourdough Stuffing with Pears and Sausage},
	description = {Sourdough bread gives the stuffing a tangier flavor than French bread, but you can use the latter in a pinch. From Cooking Light.},
	serves = {12},
	ingredients = {
	    \ingredient{8 \cu (½-inch) cubed sourdough bread (about 12 ounces)}
        \ingredient{1 pound turkey Italian sausage}
        \ingredient{Cooking spray}
        \ingredient{5 \cu chopped onion (about 2 pounds)}
        \ingredient{2 \cu chopped celery}
        \ingredient{1 \cu chopped carrot}
        \ingredient{1 (8-ounce) package presliced mushrooms}
        \ingredient{2 \cu (½-inch) cubed peeled Bartlett pear (about 2 medium)}
        \ingredient{1 ½ \tbl chopped fresh basil}
        \ingredient{2 \tsp chopped fresh tarragon}
        \ingredient{1 \tsp salt}
        \ingredient{1 ½ \cu fat-free, less-sodium chicken broth}
        \ingredient{½ \tsp freshly ground black pepper}

	},
	instructions = {
	    \instruction{Preheat oven to 425°}
		\instruction{Arrange bread in a single layer on a baking sheet. Bake at 425° for 9 minutes or until golden. Place in a large bowl.}
		\instruction{Remove casings from sausage. Heat a large nonstick skillet over medium-high heat. Coat pan with cooking spray. Add sausage, and cook for 8 minutes or until browned, stirring to crumble. Add sausage to bread cubes, tossing to combine. Set aside.}
		\instruction{Return pan to medium-high heat. Add onion, celery, and carrot; sauté 10 minutes or until onion begins to brown. Stir in mushrooms; cook 4 minutes. Stir in pear, basil, tarragon, and salt; cook 4 minutes or until pear begins to soften, stirring occasionally. Add pear mixture to bread mixture, tossing gently to combine. Stir in broth and pepper.}
		\instruction{Place bread mixture in a 13 x 9-inch baking dish coated with cooking spray; cover with foil. Bake at 425° for 20 minutes. Uncover; bake stuffing an additional 15 minutes or until top of stuffing is crisp.}
	}
}

\recipe{%
    % layout={simple},
    image={../images/recipes/spanish_rice.jpeg},
    columnratio={0.35, 0.65},
	title = {Spanish Rice},
	indexes = {Spanish Rice, Rice!Spanish Rice, Mexican cuisine, Quick meals},
	description = {Restaurant-Style Spanish Rice (aka Mexican Rice), that is easy to make and tastes like it came straight from the restaurant, and is the best side dish to any Mexican entree.},
	serves = {12},
	preptime = {5 mins},
	cookingtime = {40 mins},
	difficulty = {Beginner},
	origin = {USA},
	% tags = {Dessert, Cookies, Quick, Baked},
	% vegetarian = {yes},
	% spicy = {yes},
	extrainstructioninfo = {Place additional comments here.},
	ingredients = {
	    \ingredient{2 \cu long grain rice}
	    \ingredient{2 \tbl oil}
	    \ingredient{8 oz. tomato sauce}
	    \ingredient{6 stems cilantro}
	    \ingredient{1 \tsp salt}
	    \ingredient{1 \tsp minced garlic}
	    \ingredient{4 \cu water}
	    \ingredient{1 cube chicken flavored bouillon}
	    \ingredient{dash cumin}
	    \ingredient{dash garlic pepper}
	},
	instructions = {
	    \instruction{Heat oil in large frying pan on medium heat.}
	    \instruction{Add rice and cook until golden brown.}
	    \instruction{Add 1 chicken flavored bouillon cube to 4 cups water and microwave for 3 minutes.}
	    \instruction{When rice is brown, add water, tomato sauce, salt, garlic, cumin and garlic pepper to pan.}
	    \instruction{Stir and cover pan. Let simmer for 30-40 minutes or until all is cooked and there is no liquid left. Fluff before serving and ENJOY.}
	}
}

\recipe{%
    % layout={simple},
    image={../images/recipes/rice_pilaf.jpg},
    columnratio={0.35, 0.65},
	title = {Rice Pilaf},
	indexes = {Rice Pilaf, Rice!Rice Pilaf},
	description = {From Alton Brown},
	serves = {6},
	preptime = {15 mins},
	cookingtime = {30 mins},
	difficulty = {Beginner},
	origin = {USA},
	tags = {Dessert, Cookies, Quick, Baked},
	% vegetarian = {yes},
	% spicy = {yes},
	extrainstructioninfo = {Place additional comments here.},
	ingredients = {
        \ingredient{1 tablespoon unsalted butter}
        \ingredient{½ medium onion, finely chopped}
        \ingredient{½ medium red bell pepper, finely chopped}
        \ingredient{1 ½ \tsp kosher salt plus 2 pinches}
        \ingredient{2 cups long-grain white rice}
        \ingredient{1 pinch saffron, steeped in ¼ cup hot but not boiling water}
        \ingredient{2 ½ cups chicken broth}
        \ingredient{1 1-by-2-inch strip orange zest}
        \ingredient{2 bay leaves}
        \ingredient{½ cup peas, fresh or frozen}
        \ingredient{¼ cup golden raisins}
        \ingredient{¼ cup pistachios, chopped}
	},
	instructions = {
	    \instruction{Preheat the oven to 350.}
	    \instruction{Melt the butter in a 3-quart saucier over medium heat.}
	    \instruction{Stir in the onion, bell pepper and 2 pinches of salt.}
	    \instruction{Decrease the heat to low and sweat until the onion is translucent and aromatic but not browned, 3 to 4 minutes. Increase the heat to medium and add the rice. Cook, stirring frequently, until you smell nuts, another 3 to 4 minutes.}
	    \instruction{Add the saffron and its water, the broth, orange zest, bay leaves and the remaining 1 ½ teaspoons salt. Increase the heat and bring to a boil.}
	    \instruction{OK, now the weird part: Thoroughly wet a clean towel, kill the heat, scatter the peas on top of the rice, then place the towel across the top of the saucier. Top with the lid, then fold the towel corners up over the lid.}
	    \instruction{Transfer the saucier (towel and all) to the oven and bake 15 minutes.}
	    \instruction{Remove and rest at room temperature for 15 more minutes without opening the lid.}
	    \instruction{Fish out the orange zest and bay leaves. Turn the pilaf out onto a platter, fluff with a large fork and garnish with the raisins and pistachios. Serve family-style, right in the middle of the table.}
	}
}

%----------------------------------------------------------------------------------------
%	ENTREES CHAPTER PAGE
%----------------------------------------------------------------------------------------

\makechapterpage{
	title={Entrees},
	image={../images/book/pasta.jpg}
}

%----------------------------------------------------------------------------------------------------------


%----------------------------------------------------------------------------------------
%	RECIPE - Fajitas
%----------------------------------------------------------------------------------------

\recipe{%
	title = {Beef and Chicken Fajitas with Peppers and Onions},
	indexes = {Chicken!Beef and Chicken Fajitas with Peppers and Onions, Beef!Beef and Chicken Fajitas with Peppers and Onions, Tex-Mex!Beef and Chicken Fajitas with Peppers and Onions, Grilled!Beef and Chicken Fajitas with Peppers and Onions},
	description = {From Cooking Light Magazine.},
	serves = {4},
	preptime = {30 mins},
	cookingtime = {20 mins},
	difficulty = {Beginner},
	origin = {Tex-Mex},
	tags = {Chicken, Beef, Grilled, Tex-Mex, Maranade},
	ingredients = {
			\ingredientsection{Marinade}
			\ingredient{¼ \cu olive oil}
			\ingredient{1 \tsp lime rind,\\ grated}
			\ingredient{2 ½ \tbl lime juice}
			\ingredient{2 \tbl Worcestershire sauce}
			\ingredient{1 ½ \tsp ground cumin}
			\ingredient{1 \tsp salt}
			\ingredient{½ \tsp dried oregano}
			\ingredient{½ \tsp fresh coarse ground black pepper}
			\ingredient{3 garlic cloves,\\ minced}
			\ingredient{1 (14 ¼ ounce) can low sodium beef broth}

			\ingredientsection{Fajitas}
			\ingredient{1 (1-pound) flank steak}
			\ingredient{1 pound skinned, boned chicken breast}
			\ingredient{2 red bell peppers,\\ each cut into 12 wedges}
			\ingredient{2 green bell peppers,\\ each cut into 12 wedges}
			\ingredient{1 large  Vidalia onion,\\ cut into 16 wedges}
			\ingredient{cooking spray}
			\ingredient{16 (6-inch) flour tortillas}
			\ingredient{salsa}
			\ingredient{sour cream}
			\ingredient{cilantro}
			\ingredient{cheese}
			\ingredient{guacamole}
		},
	instructions = {
			\instructionsection{Preparing the Marinade}
			\instruction{Combine first 10 ingredients in a large bowl; set aside.}
			\instructionsection{Fajitas}
			\instruction{To prepare fajitas, trim fat from steak. Score a diamond pattern on both sides of the steak. Combine 1½ cups marinade, steak, and chicken in a large zip-top plastic bag. Seal and marinate in refrigerator 4 hours or overnight, turning occasionally. Combine remaining marinade, bell peppers, and onion in a zip-top plastic bag. Seal and marinate in refrigerator for 4 hours or overnight, turning occasionally.}
			\instruction{Prepare grill.}
			\instruction{Remove steak and chicken from bag; discard marinade. Remove vegetables from bag; reserve marinade. Place reserved marinade in a small saucepan; set aside. Place steak, chicken, and vegetables on grill rack coated with cooking spray; cook 8 minutes on each side or until desired degree of doneness.}
			\instruction{Wrap tortillas tightly in foil; place tortilla packet on grill rack the last 2 minutes of grilling time. Bring reserved marinade to a boil. Cut steak and chicken diagonally across the grain into thin slices. Place the steak, chicken, and vegetables on a serving platter; drizzle with reserved marinade.}
			\instruction{Arrange about 1 ounce steak, about 1 ounce chicken, 3 bell pepper wedges, and 1 onion wedge in a tortilla; top with 1 tablespoon salsa, about 1 teaspoon sour cream, and ½ tablespoon cilantro.
Fold sides of tortilla over filling. Garnish with cilantro sprigs, if desired. Serve immediately.}
		}
}

\recipe{%
    % layout={simple},
    % image={../images/recipes/foo.jpg},
    columnratio={0.35, 0.65},
	title = {Apple Cider-Brined Turkey with Savory Herb Gravy},
	indexes = {Pizza, Eggs, Recipes!Breakfast},
	description = {Recipe Description here.},
	serves = {12},
	preptime = {15 mins},
	cookingtime = {10 mins},
	difficulty = {Beginner},
	% origin = {USA},
	tags = {Breakfast, Pizza, Quick, Baked},
	% vegetarian = {yes},
	% spicy = {yes},
	% extrainstructioninfo = {Place additional comments here.},
	ingredients = {
	    \ingredientsection{Brine}
	    \ingredient{8 \cu apple cider}
		\ingredient{\nicefrac{2}{3} \cu kosher salt}
		\ingredient{\nicefrac{2}{3} \cu sugar}
		\ingredient{1 \tbl black peppercorns, coarsely crushed}
		\ingredient{1 \tbl whole allspice, coarsely crushed}
		\ingredient{8 1/8-inch-thick slices peeled fresh ginger}
		\ingredient{6 whole cloves}
		\ingredient{2 bay leaves}
		\ingredient{1 12-pound fresh or frozen turkey, thawed}
		\ingredient{2 oranges, quartered}
		\ingredient{6 \cu ice}
		
		\ingredientsection{Broth}
        \ingredient{4 garlic cloves}
        \ingredient{4 sage leaves}
        \ingredient{4 thyme sprigs}
        \ingredient{4 parsley sprigs}
        \ingredient{1 onion, quartered}
        \ingredient{1 (14-ounce) can fat-free, less-sodium chicken broth}
        \ingredient{2 \tbl unsalted butter, melted and divided}
        \ingredient{1 \tsp freshly ground black pepper, divided}
        \ingredient{½ \tsp salt, divided}
	},
	instructions = {
	    \instruction{To prepare brine, combine first 8 ingredients in a large saucepan; bring to a boil. Cook 5 minutes or until sugar and salt dissolve. Cool completely.}
		\instruction{Remove giblets and neck from turkey; reserve for Savory Herb Gravy. Rinse turkey with cold water; pat dry. Trim excess fat. Stuff body cavity with orange quarters. Place a turkey-sized oven bag inside a second bag to form a double thickness. Place bags in a large stockpot. Place turkey inside inner bag. Add cider mixture and ice. Secure bags with several twist ties. Refrigerate for 12 to 24 hours, turning occasionally.}
		\instruction{Remove turkey from bags, and discard brine, orange quarters, and bags. Rinse turkey with cold water; pat dry. Lift wing tips up and over back; tuck under turkey. Tie legs together with kitchen string. Place garlic, sage, thyme, parsley, onion, and broth in the bottom of roasting pan. Place roasting rack in pan. Arrange turkey, breast side down, on roasting rack. Brush turkey back with 1 tablespoon butter; sprinkle with 1/2 teaspoon pepper and 1/4 teaspoon salt. Bake at 450 degrees for 30 minutes.}
		\instruction{Reduce oven temperature to 350°}
		\instruction{Remove turkey from oven. Carefully turn turkey over (breast side up) using tongs. Brush turkey breast with 1 tablespoon butter; sprinkle with 1/2 teaspoon pepper and 1/4 teaspoon salt. Bake at 350º for 1 hour and 15 minutes or until a thermometer inserted into meaty part of thigh registers 170º (make sure not to touch bone). (Shield the turkey with foil if it browns too quickly.) Remove turkey from oven; let stand 20 minutes. Reserve pan drippings for Savory Herb Gravy. Discard skin before serving; serve with gravy.}
	}
}

\recipe{%
    columnratio={0.35, 0.65},
	title = {Savory Herb Gravy},
    ingredients = {
		\ingredient{2 \tsp vegetable oil}
        \ingredient{turkey neck and giblets}
        \ingredient{12 \cu water}
        \ingredient{6 black peppercorns}
        \ingredient{4 parsley sprigs}
        \ingredient{2 thyme sprigs}
        \ingredient{1 yellow onion, unpeeled and quartered}
        \ingredient{1 carrot, cut into 2-inch pieces}
        \ingredient{1 celery stalk, cut into 2-inch pieces}
        \ingredient{1 bay leaf}
		\ingredient{reserved turkey drippings}
        \ingredient{3 \tbl all-purpose flour}
        \ingredient{½ \tsp salt}
        \ingredient{¼ \tsp freshly ground black pepper}
	}
	instructions = {
	    \instruction{Heat oil in large saucepan over medium-high heat. Add turkey neck and giblets; cook 5 minutes, browning on all sides. Add water and next 7 ingredients (through bay leaf); bring to a boil. Reduce heat, and simmer until liquid is reduced to about 2 1/2 cups (about 1 hour). Strain through a colander over a bowl, reserving cooking liquid and turkey neck. Discard remaining solids. Chill cooking liquid completely. Skim fat from surface, and discard. Remove meat from neck; finely chop meat. Discard neck bone. Add neck meat to cooking liquid.}
		\instruction{Strain the reserved turkey drippings through a colander over a shallow bowl; discard solids. Place strained drippings in freezer for 20 minutes. Skim fat from surface; discard.}
		\instruction{Place flour in a medium saucepan; add 1/4 cup cooking liquid, stirring with a whisk until smooth. Add remaining cooking liquid, turkey drippings, salt, and pepper; bring to a boil, stirring frequently. Reduce heat; simmer 5 minutes or until slightly thickened.}
		
	}
}
\recipe{%
    % layout={simple},
    image={../images/recipes/herbed_chicken_parmesan.jpg},
    columnratio={0.35, 0.65},
	title = {Herbed Chicken Parmesan},
	indexes = {Herbed Chicken Parmesan, Italian Cuisine, Chicken, Recipes!Entree},
	description = {This lightened version of an Italian favorite loses some of the fat but none of the flavor. Cooking Light.},
	serves = {4},
	preptime = {15 mins},
	cookingtime = {20 mins},
	difficulty = {Beginner},
	origin = {Italy},
	tags = {Italian, Quick, Baked},
	% vegetarian = {yes},
	% spicy = {yes},
	extrainstructioninfo = {Place additional comments here.},
	ingredients = {
	    \ingredient{\nicefrac{1}{3} \cu (1 ½ ounces) grated fresh Parmesan cheese, divided}
	    \ingredient{¼ \cu dry breadcrumbs}
	    \ingredient{1 \tbl minced fresh parsley}
	    \ingredient{½ \tsp dried basil}
	    \ingredient{¼ \tsp salt, divided}
	    \ingredient{1 large egg white, lightly beaten}
	    \ingredient{1 pound chicken breast tenders}
	    \ingredient{1 \tbl butter}
	    \ingredient{1 ½ \cu bottled fat-free tomato-basil pasta sauce (such as Muir Glen Organic)}
	    \ingredient{2 \tsp balsamic vinegar}
	    \ingredient{¼ \tsp black pepper}
	    \ingredient{\nicefrac{1}{3} \cu (1 ½ ounces) shredded provolone cheese}
		\ingredient{orzo}
		\ingredient{broccoli}
		\ingredient{lemon zest}
		\ingredient{garlic}
	},
	instructions = {
	    \instruction{Preheat broiler.}
	    \instruction{Combine 2 tablespoons of Parmesan, breadcrumbs, parsley, basil, and 1/8 teaspoon salt in a shallow dish. Place egg white in a shallow dish. Dip each chicken tender in egg white; dredge in the breadcrumb mixture. Melt butter in a large nonstick skillet over medium-high heat. Add chicken; cook 3 minutes on each side or until done. Set aside.}
	    \instruction{Combine 1/8 teaspoon salt, pasta sauce, vinegar, and pepper in a microwave-safe bowl. Cover with plastic wrap; vent. Microwave sauce mixture at HIGH 2 minutes or until thoroughly heated. Pour the sauce over chicken in pan. Sprinkle evenly with the remaining Parmesan and provolone cheese. Wrap handle of pan with foil, and broil 2 minutes or until the cheese melts.}
	}
}

\recipe{%
    layout={simple},
    image={../images/recipes/pulled_chicken.jpg},
    columnratio={0.25, 0.75},
	title = {Pulled Chicken Sandwiches},
	indexes = {Sandwich, Chicken, Recipes!Breakfast},
	description = {Dinner guests are guaranteed to be impressed with this deceptively easy Pulled Chicken Sandwich recipe, which includes a seven-ingredient rub and a simple 15-minute sauce that comes together while the chicken grills. Serve on buns, over fresh greens, or on top of a baked potato for a filling dinner. Cooking Light},
	serves = {8},
	preptime = {20 mins},
	cookingtime = {20 mins},
	difficulty = {Beginner},
	% origin = {USA},
	tags = {Sandwich, Chicken},
	% vegetarian = {yes},
	% spicy = {yes},
	extrainstructioninfo = {The chicken and sauce can be made up to two days ahead and stored in the refrigerator.},
	ingredients = {
	    \begin{minipage}[t]{.32\textwidth}
	    \ingredientsection{Chicken}
	    \ingredient{2 \tbl dark brown sugar}
	    \ingredient{1 \tsp paprika}
	    \ingredient{¾ \tsp ground cumin}
	    \ingredient{½ \tsp ground chipotle chile pepper}
	    \ingredient{¼ \tsp ground ginger}
	    \ingredient{\nicefrac{1}{8} \tsp salt}
	    \ingredient{2 pounds skinless, boneless chicken thighs}
	    \ingredient{Cooking spray}
		\end{minipage}
		\hfill
		\begin{minipage}[t]{.32\textwidth}
	    \ingredientsection{Sauce}
	    \ingredient{2 \tsp canola oil}
	    \ingredient{½ \cu finely chopped onion}
	    \ingredient{1 \tbl dark brown sugar}
	    \ingredient{1 \tsp chili powder}
	    \ingredient{½ \tsp garlic powder}
	    \ingredient{½ \tsp dry mustard}
	    \ingredient{¼ \tsp ground allspice}
	    \ingredient{\nicefrac{1}{8} \tsp ground red pepper}
	    \ingredient{1 \cu ketchup}
	    \ingredient{2 \tbl cider vinegar}
		\end{minipage}
		\hfill
		\begin{minipage}[t]{.32\textwidth}
	    \ingredientsection{Remaining ingredients}
	    \ingredient{8 (1 ½-ounce) hamburger buns, toasted}
	    \ingredient{16 hamburger dill chips}
        \end{minipage}
	},
	instructions = {
	    \instruction{Preheat grill to medium-high heat.}
	    \instruction{To prepare chicken, combine first 6 ingredients; rub evenly over chicken. Place chicken on a grill rack coated with cooking spray; cover and grill 15 minutes or until a thermometer registers 180°, turning occasionally. Let stand for 5 minutes. Shred with 2 forks.}
	    \instruction{To prepare sauce, while chicken grills, heat canola oil in a medium saucepan over medium heat. Add onion; cook for 5 minutes or until tender, stirring occasionally. Stir in 1 tablespoon sugar and next 5 ingredients (through ground red pepper); cook 30 seconds. Stir in ketchup and vinegar; bring to a boil. Reduce heat, and simmer 10 minutes or until slightly thickened, stirring occasionally. Stir in chicken; cook 2 minutes.}
	    \instruction{Place 1/3 cup chicken mixture on bottom half of each bun; top each with 2 pickle chips and top of bun.}
	}
}

\recipe{%
    % layout={simple},
    image={../images/recipes/beef_daube_provencal.jpg},
    columnratio={0.35, 0.65},
	title = {Beef Daube Procençal},
	indexes = {Beef, Stew, Recipes!Entree, French Cuisine},
	description = {Best Beef Recipe. This classic French braised beef, red wine, and vegetable stew is simple and delicious. It stands above all of our other beef recipes because it offers the homey comfort and convenience of pot roast yet is versatile and sophisticated enough for entertaining. Garnish with chopped fresh thyme.},
	serves = {12},
	preptime = {15 mins},
	cookingtime = {10 mins},
	difficulty = {Beginner},
	origin = {USA},
	tags = {Breakfast, Pizza, Quick, Baked},
	% vegetarian = {yes},
	% spicy = {yes},
	extrainstructioninfo = {To make in a slow cooker, prepare through Step Place beef mixture in an electric slow cooker. Cover and cook on HIGH for 5 hours.},
	ingredients = {
	    \ingredient{2 \tsp olive oil}
	    \ingredient{12 garlic cloves, crushed}
	    \ingredient{1 (2-pound) boneless chuck roast, trimmed and cut into 2-inch cubes}
	    \ingredient{1 ½ \tsp salt, divided}
	    \ingredient{½ \tsp freshly ground black pepper, divided}
	    \ingredient{1 \cu red wine}
	    \ingredient{2 \cu chopped carrot}
	    \ingredient{1 ½ \cu chopped onion}
	    \ingredient{½ cup less-sodium beef broth}
	    \ingredient{1 \tbl tomato paste}
	    \ingredient{1 \tsp chopped fresh rosemary}
	    \ingredient{1 \tsp chopped fresh thyme}
	    \ingredient{Dash of ground cloves}
	    \ingredient{1 (14.5-ounce) can diced tomatoes, undrained}
	    \ingredient{1 bay leaf}
	    \ingredient{3 \cu hot cooked medium egg noodles (about 4 cups uncooked noodles)}
	    \ingredient{Chopped fresh thyme (optional)}
	},
	instructions = {
	    \instruction{Preheat oven to 300°.}
		\instruction{Heat olive oil in a small Dutch oven over low heat. Add garlic to pan; cook for 5 minutes or until garlic is fragrant, stirring occasionally. Remove garlic with a slotted spoon; set aside. Increase heat to medium-high. Add beef to pan. Sprinkle beef with ½ teaspoon salt and ¼ teaspoon black pepper. Cook 5 minutes, browning on all sides. Remove beef from pan. Add wine to pan, and bring to a boil, scraping pan to loosen browned bits. Add garlic, beef, remaining 1 teaspoon salt, remaining ¼ teaspoon pepper, carrot, and next 8 ingredients (through bay leaf) to pan; bring to a boil.}
		\instruction{Cover and bake at 300° for 2 ½ hours or until beef is tender. Discard bay leaf. Serve over noodles. Garnish with chopped fresh thyme, if desired.}
	}
}

\recipe{%
    % layout={simple},
    image={../images/recipes/beef_rendang.jpg},
    columnratio={0.35, 0.65},
	title = {Beef Rendang},
	indexes = {Beef, Stew, Recipes!Entree, Malasian Cuisine},
	description = {This rich Malay curry features aromatic lemongrass, garlic, ginger, and cinnamon. Make sure the beef mixture cooks at a low simmer so the sauce doesn't scorch and the meat slowly becomes tender. If you can't find unsweetened coconut, use sweetened flaked coconut and omit the added sugar.},
	serves = {6},
	preptime = {15 mins},
	cookingtime = {10 mins},
	difficulty = {Beginner},
	origin = {USA},
	tags = {Breakfast, Pizza, Quick, Baked},
	% vegetarian = {yes},
	% spicy = {yes},
	extrainstructioninfo = {Place additional comments here.},
	ingredients = {
	    \ingredient{½ \cu chopped shallots}
        \ingredient{1/3 \cu thinly sliced peeled ginger}
        \ingredient{1 ½ \tbl minced garlic (about 5 cloves)}
        \ingredient{2 \tbl chili garlic sauce (such as Lee Kum Kee)}
        \ingredient{1 ½ \tsp ground turmeric}
        \ingredient{1 ¼ \tsp salt}
        \ingredient{¼ \tsp ground cinnamon}
        \ingredient{6 whole cloves}
        \ingredient{1 to 2 serrano chiles, chopped}
        \ingredient{1 (14-ounce) can light coconut milk, divided}
        \ingredient{2/3 \cu flaked unsweetened coconut, toasted}
        \ingredient{1 \tsp grated lime rind}
        \ingredient{2 \tbl fresh lime juice}
        \ingredient{2 \tsp sugar}
        \ingredient{2 (3-inch) fresh lemongrass stalks, crushed}
        \ingredient{2 pounds boneless chuck roast, trimmed and cut into 1-inch cubes}
        \ingredient{1 (14-ounce) can fat-free, less-sodium chicken broth}
        \ingredient{4 \cu hot cooked basmati rice}

	},
	instructions = {
	    \instruction{Place first 9 ingredients in a food processor or mini chopper. Add ¼ cup coconut milk; process until smooth. Spoon mixture into a bowl; set aside.}
		\instruction{Place 3 tablespoons coconut milk and flaked coconut in food processor; process until a smooth paste forms.}
		\instruction{Heat a large saucepan over medium-high heat. Add shallot mixture; cook 1 minute or until fragrant, stirring constantly. Stir in remaining coconut milk, rind, and next 5 ingredients (through broth); bring to a boil. Cover, reduce heat to medium-low, and simmer 90 minutes or until beef is tender, stirring occasionally. Discard lemongrass. Stir in flaked coconut mixture; simmer 10 minutes or until liquid almost evaporates. Serve over rice.}
	}
}

\recipe{%
    % layout={simple},
    image={../images/recipes/meat_sauce.jpeg},
    columnratio={0.35, 0.65},
	title = {Slow Simmered Meat Sauce},
	indexes = {Pasta, Sauce, Recipes!Entree, Italian Cuisine},
	description = {Mafaldine is a flat noodle with ruffled edges. You can substitute spaghetti. From Cooking Light.},
	serves = {8},
	preptime = {30 mins},
	cookingtime = {8 hr},
	difficulty = {Beginner},
	origin = {USA},
	tags = {Sauce, Pastz, Slow-Cooker},
	spicy = {yes},
	% extrainstructioninfo = {Place additional comments here.},
	ingredients = {
	    \ingredient{1 \tbl olive oil}
	    \ingredient{2 \cu chopped onion}
	    \ingredient{1 \cu chopped carrot}
	    \ingredient{6 garlic cloves, minced}
	    \ingredient{2 (4-ounce) links hot Italian sausage, casings removed}
	    \ingredient{1 pound ground sirloin}
	    \ingredient{½ \cu kalamata olives, pitted and sliced}
	    \ingredient{¼ \cu no-salt-added tomato paste}
	    \ingredient{1 ½ \tsp sugar}
	    \ingredient{1 \tsp kosher salt}
	    \ingredient{½ \tsp crushed red pepper}
	    \ingredient{1 (28-ounce) can no-salt-added crushed tomatoes, undrained}
	    \ingredient{1 \cu no-salt-added tomato sauce}
	    \ingredient{1 \tbl chopped fresh oregano}
	    \ingredient{16 ounces uncooked mafaldine pasta}
	    \ingredient{½ \cu torn fresh basil}
	    \ingredient{3 ounces shaved fresh Parmigiano-Reggiano cheese}
	},
	instructions = {
	    \instruction{Heat a large skillet over medium-high heat. Add oil to pan; swirl to coat. Add onion and carrot to pan; sauté 4 minutes, stirring occasionally. Add garlic; sauté 1 minute, stirring constantly. Place vegetable mixture in a 6-quart slow cooker. Add sausage and beef to skillet; sauté 6 minutes or until browned, stirring to crumble. Remove beef mixture from skillet using a slotted spoon. Place beef mixture on a double layer of paper towels; drain. Add beef mixture to slow cooker. Stir olives and next 6 ingredients (through tomato sauce) into slow cooker. Cover and cook on LOW 8 hours. Stir in oregano.}
		\instruction{Prepare pasta according to package directions, omitting salt and fat. Serve sauce with hot cooked pasta; top with basil and cheese.}
	}
}

\recipe{%
    % layout={simple},
    % image={../images/recipes/foo.png},
    columnratio={0.35, 0.65},
	title = {Gnocchi Bolognese},
	indexes = {Beef, Stew, Recipes!Entree, Italian Cuisine},
	description = {Recipe Description here.},
	serves = {12},
	preptime = {15 mins},
	cookingtime = {10 mins},
	difficulty = {Beginner},
	origin = {USA},
	tags = {Breakfast, Pizza, Quick, Baked},
	% vegetarian = {yes},
	% spicy = {yes},
	extrainstructioninfo = {Place additional comments here.},
	ingredients = {
	    \ingredientsection{Section 1}
	    \ingredient{First Ingredient}
	},
	instructions = {
	    \instructionsection{First Section}
	    \instruction{Place the text for the first step here}
	}
}


%----------------------------------------------------------------------------------------
%	DESSERTS CHAPTER PAGE
%----------------------------------------------------------------------------------------

\makechapterpage{
	title={Desserts},
	image={../images/book/dessert.jpg},
	layout={right}
}

\recipe{%
    % layout={simple},
    % image={../images/recipes/foo.png},
    columnratio={0.35, 0.65},
	title = {Chocolate Cake with Mascarpone Frosting},
	% indexes = {Pizza, Eggs, Recipes!Breakfast},
	description = {Recipe Description here.},
	serves = {12},
	preptime = {15 mins},
	cookingtime = {10 mins},
	difficulty = {Beginner},
	origin = {USA},
	tags = {Breakfast, Pizza, Quick, Baked},
	vegetarian = {yes},
	% spicy = {yes},
	extrainstructioninfo = {Place additional comments here.},
	ingredients = {
	    \ingredientsection{Section 1}
	    \ingredient{First Ingredient}
	},
	instructions = {
	    \instructionsection{First Section}
	    \instruction{Place the text for the first step here}
	}
}

\recipe{%
    % layout={simple},
    % image={../images/recipes/foo.png},
    columnratio={0.35, 0.65},
	title = {Strawberry Cake},
	indexes = {Pizza, Eggs, Recipes!Breakfast},
	description = {Recipe Description here.},
	serves = {12},
	preptime = {15 mins},
	cookingtime = {10 mins},
	difficulty = {Beginner},
	origin = {USA},
	tags = {Breakfast, Pizza, Quick, Baked},
	% vegetarian = {yes},
	% spicy = {yes},
	extrainstructioninfo = {Place additional comments here.},
	ingredients = {
	    \ingredientsection{Section 1}
	    \ingredient{First Ingredient}
	},
	instructions = {
	    \instructionsection{First Section}
	    \instruction{Place the text for the first step here}
	}
}

%----------------------------------------------------------------------------------------
%	CONVERSION TABLE
%----------------------------------------------------------------------------------------

\makeconversionpage{
	title={Conversion Tables}
}

%----------------------------------------------------------------------------------------
%	INDEX
%----------------------------------------------------------------------------------------

\printindex

%----------------------------------------------------------------------------------------
%	BACK COVER
%----------------------------------------------------------------------------------------

\makebackcoverpage{
	topcontent={
		{\fontsize{24pt}{28pt}\sourcesanspro\bfseries\selectfont\color{white}\MakeUppercase{About This Book}}\par
		\vspace{0.02\textheight}
		This cookbook represents a collection of cherished recipes passed down through generations, each one telling a story of family gatherings, holiday celebrations, and everyday moments made special by the food we share. From simple comfort foods to elaborate feasts, these recipes have been tested, refined, and perfected over countless meals.\newline\newline
		Whether you're a seasoned cook or just beginning your culinary journey, we hope these recipes bring joy, inspiration, and delicious results to your kitchen.
	},
	image={../images/book/back-cover.jpg},
	imageopacity={0.8},
	imageposition={right},
	columnratio={0.5,0.5},
	verticalsplit={0.5},
	bottomcontent={
		\textbf{From Our Kitchen to Yours:}\par
		\vspace{0.01\textheight}
		Start your morning right with our fluffy \textbf{Banana Pancakes}—a family favorite that's both simple and satisfying. For a special weekend treat, you simply must try our \textbf{Classic French Toast}, golden and perfectly crisp.\newline\newline
		When it comes to main courses, our \textbf{Spaghetti Bolognese} has been a Sunday dinner tradition for decades, while the \textbf{Lemon Herb Grilled Salmon} brings elegance to any weeknight meal. And don't miss our family's favorite dessert—the \textbf{Classic Tiramisu} that has graced countless celebrations and always leaves guests asking for the recipe.\newline\newline
		\textit{Each recipe includes detailed instructions, ingredient lists, and helpful tips to ensure your success in the kitchen.}
	},
	isbn={978-0-123456-78-9},
	publisher={Published by Your Publisher Name},
	copyright={© 2025 All rights reserved.},
	textcolor={white},
	bgcolor={darkgrey},
	divider={true},
	barcodeplaceholder={true}
}

\end{document}
