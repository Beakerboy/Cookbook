\documentclass[
	letterpaper,
	10pt,
	twoside
]{CookBook}

\begin{document}

\makecoverpage{
	title={My Cookbook},
	subtitle={Our Favorite Recipes},
	author={Kevin Nowaczyk},
	titlefontsize={\fontsize{36pt}{38pt}},
	subtitlefontsize={\fontsize{24pt}{26pt}},
	image={../images/book/cover.jpg},
	opacity={0.6},
	bgcolor={darkgrey},
	textcolor={white},
	shadowoffset={0.05cm}
}

%----------------------------------------------------------------------------------------
%	PREFACE
%----------------------------------------------------------------------------------------

\makeprefacepage{
	title={Preface},
	text={Welcome to this collection of recipes, gathered from three generations of family cooking and culinary adventures around the world. Each recipe tells a story—some passed down through handwritten notes on yellowed paper, others discovered during travels through bustling markets and quiet countryside kitchens.\newline

	This cookbook is more than just a collection of instructions and ingredients. It's a celebration of the joy that comes from creating something delicious, the warmth of sharing a meal with loved ones, and the memories that form around the dinner table. From simple weekday breakfasts to elaborate weekend feasts, these recipes have been tested, adjusted, and perfected over countless meals.\newline

	Whether you're a beginner just learning your way around the kitchen or an experienced cook looking for new inspiration, I hope these recipes bring as much happiness to your table as they have brought to ours. Don't be afraid to make them your own—the best recipes are the ones that evolve with each telling.\newline

	Happy cooking!},
	layout={single},
	image={../images/book/preface.jpg}
}

%----------------------------------------------------------------------------------------
%	TABLE OF CONTENTS
%----------------------------------------------------------------------------------------

\maketoc

%----------------------------------------------------------------------------------------
%	BREAKFAST CHAPTER PAGE
%----------------------------------------------------------------------------------------

\makechapterpage{
	title={Breakfast},
	bgcolor={paleorange},
	image={../images/book/breakfast.jpg},
	layout={right}
}


%----------------------------------------------------------------------------------------------------------


%----------------------------------------------------------------------------------------
%	RECIPE EXAMPLE - Breakfast Sausage Casserole
%----------------------------------------------------------------------------------------

\recipe{%
	layout={simple},
	imageheight={0.25\paperheight},
	imageoverlayspace={0.2\paperheight},
	title = {Breakfast Sausage Casserole},
	description = {This satisfying recipe is perfect to make for weekend guests. Assemble and refrigerate the casserole the night before, and just pop it in the oven the next morning. Look for turkey sausage near other breakfast-style sausage in the frozen foods section.},
	serves = {8},
	preptime = {15 mins},
	cookingtime = {65 mins},
	difficulty = {Beginner},
	origin = {USA},
	tags = {Breakfast},
	indexes = {Breakfast Casserole, Recipes!Breakfast, American cuisine},
	ingredients = {
	    \ingredient{1 (16-ounce) package sage sausage}
		},
	instructions = {
	    \instruction{Heat a large nonstick skillet over medium-high heat. Coat pan with cooking spray. Add sausage to pan, cook 5 minutes or until browned, stirring well to crumble.}
	}
}


%----------------------------------------------------------------------------------------------------------


%----------------------------------------------------------------------------------------
%	RECIPE EXAMPLE - Bacon, Ham, and Egg Hash (Columns Layout, No Image)
%----------------------------------------------------------------------------------------

\recipe{%
    image = {../images/recipes/IMG_9660.jpeg},
	title = {Bacon, Ham, and Egg Hash},
	indexes = {Hash, Recipes!Breakfast, Potato, Eggs, Quick meals},
	description = {Golden, crispy French toast with a custardy center. A weekend breakfast favorite that's easy to make yet feels special. Perfect with maple syrup and fresh berries.},
	serves = {4},
	preptime = {10 mins},
	cookingtime = {15 mins},
	difficulty = {Beginner},
	origin = {USA},
	tags = {Breakfast, Quick},
	ingredients = {
			\ingredientsection{Custard Mixture}
			\ingredient{4 large eggs}
			\ingredient{1 cup whole milk}
			\ingredient{2 tbsp granulated sugar}
			\ingredient{1 tsp vanilla extract}
			\ingredient{1 tsp ground cinnamon}
			\ingredient{1/4 tsp ground nutmeg}
			\ingredient{Pinch of salt}

			\ingredientsection{Cooking and Serving}
			\ingredient{8 thick slices bread (brioche, challah, or white bread)\note{Day-old bread works best as it absorbs the custard mixture without falling apart. Brioche or challah bread creates the most luxurious French toast, but any thick-sliced bread will work.}}
			\ingredient{2-3 tbsp butter for cooking}
			\ingredient{Maple syrup, for serving}
			\ingredient{Fresh berries, for serving}
			\ingredient{Powdered sugar for dusting (optional)}
		},
	instructions ={
			\instruction{In a shallow bowl or baking dish, whisk together eggs, milk, sugar, vanilla extract, cinnamon, nutmeg, and salt until well combined.}
			\instruction{Heat a large skillet or griddle over medium heat and add 1 tablespoon of butter.}
			\instruction{Dip each bread slice into the custard mixture, allowing it to soak for about 5 seconds per side. Don't oversoak or the bread will fall apart.}
			\instruction{Place soaked bread slices in the heated skillet. Cook for 2-3 minutes per side until golden brown and crispy on the outside.}
			\instruction{Transfer cooked French toast to a plate and keep warm. Repeat with remaining bread slices, adding more butter to the pan as needed.}
			\instruction{Serve immediately topped with maple syrup, fresh berries, and a dusting of powdered sugar if desired.}
		}
}

\makeimagepage{
	image={../images/recipes/french-toast.jpg},
	caption = {Classic French Toast},
}

%----------------------------------------------------------------------------------------------------------


%----------------------------------------------------------------------------------------
%	APPETIZERS CHAPTER PAGE
%----------------------------------------------------------------------------------------

\makechapterpage{
	title={Appetizers},
	image={../images/book/salad.jpg}
}

%----------------------------------------------------------------------------------------------------------


%----------------------------------------------------------------------------------------
%	RECIPE - Esquites (Columns Layout)
%----------------------------------------------------------------------------------------

\recipe{%
    image = {../images/recipes/IMG_9658.jpeg},
	title = {Mexican Street Corn (Esquites)},
	indexes = {Recipes!Appetizers, Corn, Vegetarian recipes, Mexican cuisine},
	description = {From SeriousEats},
	serves = {4},
	preptime = {5 mins},
	cookingtime = {15 mins},
	difficulty = {Beginner},
	origin = {Mexico},
	tags = {Mexican, Quick, Vegetarian},
	vegetarian = {yes},
	ingredients = {
			\ingredient{2 tablespoon vegtable oil}
			\ingredient{4 ears fresh corn (about 3 cups) or 2 cans}
			\ingredient{kosher salt}
			\ingredient{2 ounces fet or corija cheese, finely crumbled}
			\ingredient{1/2 cup finely sliced scallions}
			\ingredient{1/2 cup cilantro, finely chopped}
			\ingredient{1 jalapeño pepper, finely chopped }
			\ingredient{2 tablespoons mayonnaise}
			\ingredient{1 tablespoon fresh lime juice}
			\ingredient{chili powder to taste}
		},
	instructions ={
			\instruction{Heat oil in a large nonstick skillet or wok over high heat until shimmering. Add corn kernels, season to taste with salt, toss once or twice, and cook without moving until charred on one side, about 2 minutes. Toss corn, stir, and repeat until charred on second side, about 2 minutes longer. Continue tossing and charring until corn is well charred all over, about 10 minutes total. Transfer to a large bowl.}
			\instruction{Add cheese, scallions, cilantro, jalapeño, garlic, mayonnaise, lime juice, and chile powder and toss to combine. Taste and adjust seasoning with salt and more chile powder to taste. Serve immediately.}
		}
}

\recipe{%
	title = {Broccoli, Grape, and Pasta Salad},
	description = {If you're a broccoli salad fan, you'll love the combination of these colorful ingredients. Cook the pasta al dente so it's firm enough to hold its own when tossed with the tangy-sweet salad dressing.},
	ingredients = {
	    \ingredient{1 cup chopped pecans}
	},
	instructions = {
	    \instruction{Preheat oven to 350°. Bake pecans in a single layer in a shallow pan 5 to 7 minutes or until lightly toasted and fragrant, stirring halfway through.}
	}
}

\recipe{%
	title = {Roasted Potato Salad with Mistard Dressing},
	description = {This tangy side dish with sweet onions and honey pairs beautifully with burgers or steak. This salad is best chilled but can stay at room temperature for up to two hours. You can also use sweet-hot mustard in place of the Dijon and honey for a zesty flavor.},
	serves = {8},
	ingredients = {
	    \ingredient{3 pounds small red potatoes, cut into 1-inch pieces}
	},
	instructions = {
	    \instruction{Preheat oven to 400°}
	}
}

\recipe{%
	title = {Sourdough Stuffing with Pears and Sausage},
	description = {Sourdough bread gives the stuffing a tangier flavor than French bread, but you can use the latter in a pinch.},
	serves = {12},
	ingredients = {
	    \ingredient{8 cups (½-inch) cubed sourdough bread (about 12 ounces)}
	},
	instructions = {
	    \instruction{Preheat oven to 425°}
	}
}
%----------------------------------------------------------------------------------------------------------


%----------------------------------------------------------------------------------------
%	RECIPE EXAMPLE - CAESAR SALAD (Simple Layout, No Image)
%----------------------------------------------------------------------------------------

\makeimagepage{
	image={../images/recipes/caesar-salad.jpg},
	caption = {Caesar Salad},
}

\recipe{%
	layout={simple},
	title = {Caesar Salad},
	indexes = {Caesar Salad, Salad, Recipes!Appetizers, Recipes!Salads, Romaine lettuce, Parmesan, Anchovies},
	description = {The classic Caesar salad with crispy romaine lettuce, homemade dressing, and crunchy croutons. A timeless favorite that never goes out of style.},
	serves = {4},
	preptime = {20 mins},
	cookingtime = {10 mins},
	difficulty = {Intermediate},
	origin = {Mexico/USA},
	tags = {Salad, Classic, Italian, Make-Ahead},
	ingredients = {
			\ingredientsection{Dressing}
			\ingredient{2 garlic cloves, minced}
			\ingredient{2 anchovy fillets, minced}
			\ingredient{2 tbsp fresh lemon juice}
			\ingredient{1 tsp Dijon mustard}
			\ingredient{1 tsp Worcestershire sauce}
			\ingredient{1 cup mayonnaise}
			\ingredient{1/2 cup freshly grated Parmesan\note{For the best flavor, use freshly grated Parmesan cheese. The dressing can be made up to 3 days ahead and stored in the refrigerator.}}
			\ingredient{Salt and black pepper}

			\ingredientsection{Salad}
			\ingredient{2 large heads romaine lettuce, chopped}
			\ingredient{1 cup homemade or store-bought croutons}
			\ingredient{1/2 cup shaved Parmesan cheese}
			\ingredient{Freshly cracked black pepper}
		},
	instructions ={
			\instruction{In a medium bowl, whisk together garlic, anchovies, lemon juice, Dijon mustard, and Worcestershire sauce.}
			\instruction{Slowly whisk in mayonnaise until well combined and smooth.}
			\instruction{Stir in grated Parmesan cheese. Season with salt and pepper to taste. Refrigerate until ready to use.}
			\instruction{Place chopped romaine lettuce in a large serving bowl.}
			\instruction{Pour desired amount of dressing over the lettuce and toss well to coat evenly.}
			\instruction{Add croutons and toss again gently.}
			\instruction{Top with shaved Parmesan cheese and freshly cracked black pepper. Serve immediately.}
		}
}



%----------------------------------------------------------------------------------------------------------


%----------------------------------------------------------------------------------------
%	RECIPE EXAMPLE - CAPRESE SALAD (Columns Layout, No Image)
%----------------------------------------------------------------------------------------

\recipe{%
image={../images/recipes/caprese-salad.jpg},
imageposition={bottom},
imageheight={0.4\paperheight},
title = {Caprese Salad},
indexes = {Caprese Salad, Salad, Recipes!Appetizers, Recipes!Salads, Mozzarella, Tomatoes, Basil, Vegetarian recipes, Italian cuisine, No-cook meals, Gluten-free recipes},
description = {A beautiful and refreshing Italian salad showcasing the colors of the Italian flag. The quality of ingredients is key to this simple yet elegant dish.},
serves = {4},
preptime = {10 mins},
difficulty = {Beginner},
	origin = {Italy},
	tags = {Italian, No-Cook, Gluten-Free, Vegetarian},
	vegetarian = {yes},
	ingredients = {
		\ingredient{4 large ripe tomatoes, sliced 1/4 inch thick\note{Use the best quality tomatoes you can find—heirloom or vine-ripened varieties work beautifully.}}
		\ingredient{16 oz fresh mozzarella cheese, sliced 1/4 inch thick\note{Fresh mozzarella di bufala is traditional and highly recommended for the most authentic flavor.}}
		\ingredient{1 cup fresh basil leaves}
		\ingredient{3 tbsp extra virgin olive oil}
		\ingredient{2 tbsp balsamic vinegar or glaze}
		\ingredient{Flaky sea salt}
		\ingredient{Freshly ground black pepper}
	},
instructions ={
		\instruction{Arrange tomato and mozzarella slices on a large serving platter, alternating and slightly overlapping them.}
		\instruction{Tuck fresh basil leaves between the tomato and mozzarella slices.}
		\instruction{Drizzle generously with extra virgin olive oil and balsamic vinegar.}
		\instruction{Sprinkle with flaky sea salt and freshly ground black pepper.}
		\instruction{Let stand at room temperature for 5-10 minutes before serving to allow the flavors to develop.}
	}
}


%----------------------------------------------------------------------------------------------------------


%----------------------------------------------------------------------------------------
%	RECIPE EXAMPLE - SPICY CHICKEN WINGS (Simple Layout with Image)
%----------------------------------------------------------------------------------------

\recipe{%
	layout={simple},
	image={../images/recipes/buffalo-chicken-wings.jpg},
	imageheight={0.4\paperheight},
	imageoverlayspace={0.35\paperheight},
	title = {Buffalo Chicken Wings},
	indexes = {Buffalo Chicken Wings, Chicken Wings, Recipes!Appetizers, Chicken, Spicy recipes, American cuisine, Baked dishes},
	description = {Crispy, spicy chicken wings with a tangy buffalo sauce. Perfect for game day or any gathering. These wings are baked for a healthier twist on the classic.},
	serves = {4-6},
	preptime = {15 mins},
	cookingtime = {45 mins},
	difficulty = {Intermediate},
	origin = {USA},
	tags = {Spicy, Chicken, Baked, Appetizer, Game-Day},
	spicy = {yes},
	ingredients = {
			\ingredientsection{Wings}
			\ingredient{3 lbs chicken wings, separated at joints, tips removed\note{For extra crispy wings, pat them completely dry before seasoning. Serve with celery sticks and blue cheese or ranch dressing.}}
			\ingredient{1 tbsp baking powder}
			\ingredient{1 tsp salt}
			\ingredient{1 tsp garlic powder}
			\ingredient{1/2 tsp black pepper}

			\ingredientsection{Buffalo Sauce}
			\ingredient{1/2 cup hot sauce (Frank's RedHot preferred)}
			\ingredient{1/3 cup unsalted butter, melted}
			\ingredient{1 tbsp white vinegar}
			\ingredient{1/4 tsp Worcestershire sauce}
			\ingredient{1/4 tsp cayenne pepper (optional, for extra heat)}
		},
	instructions ={
			\instruction{Preheat oven to 250°F (120°C). Line a large baking sheet with aluminum foil and place a wire rack on top.}
			\instruction{Pat the chicken wings completely dry with paper towels. This is crucial for crispy skin.}
			\instruction{In a large bowl, combine baking powder, salt, garlic powder, and black pepper. Add wings and toss until evenly coated.}
			\instruction{Arrange wings on the wire rack in a single layer, making sure they don't touch.\note{Proper spacing ensures even cooking and maximum crispiness.}}
			\instruction{Bake at 250°F for 30 minutes. Increase temperature to 425°F (220°C) and bake for an additional 40-45 minutes, flipping halfway through, until golden brown and crispy.}
			\instruction{While wings are baking, prepare the buffalo sauce by whisking together hot sauce, melted butter, vinegar, Worcestershire sauce, and cayenne pepper in a small bowl.}
			\instruction{Transfer cooked wings to a large bowl, pour buffalo sauce over them, and toss to coat evenly. Serve immediately.}
		}
}


%----------------------------------------------------------------------------------------------------------

%----------------------------------------------------------------------------------------
%	ENTREES CHAPTER PAGE
%----------------------------------------------------------------------------------------

\makechapterpage{
	title={Entrees},
	image={../images/book/pasta.jpg}
}

%----------------------------------------------------------------------------------------------------------


%----------------------------------------------------------------------------------------
%	RECIPE - Fajitas
%----------------------------------------------------------------------------------------

\recipe{%
	title = {Beef and Chicken Fajitas with Peppers and Onions},
	indexes = {Spaghetti Bolognese, Bolognese, Recipes!Entrees, Recipes!Pasta, Pasta, Beef, Tomatoes, Italian cuisine, Make-ahead meals},
	description = {Bolognese sauce, known in Italian as ragú alla Bolognese, is a meat-based sauce originating from Bologna, Italy. In Italian cuisine, it is customarily used to dress ``tagliatelle al ragú" and to prepare ``lasagne alla bolognese".},
	serves = {4},
	preptime = {25 mins},
	cookingtime = {40 mins},
	difficulty = {Beginner},
	origin = {Italy},
	tags = {Pasta, Meat, Italian, Bolognese, Make-Ahead},
	ingredients = {
			\ingredientsection{Marinade}
			\ingredient{1⁄4 cup olive oil}
			\ingredient{1 teaspoon lime rind,\\ grated}
			\ingredient{2 1⁄2 tablespoons lime juice}
			\ingredient{2 tablespoons Worcestershire sauce}
			\ingredient{1 1⁄2 teaspoons ground cumin}
			\ingredient{1 teaspoon salt}
			\ingredient{1⁄2 teaspoon dried oregano}
			\ingredient{1⁄2 teaspoon fresh coarse ground black pepper}
			\ingredient{3 garlic cloves,\\ minced}
			\ingredient{1 (14 1/4 ounce) can low sodium beef broth}

			\ingredientsection{Fajitas}
			\ingredient{1 (1 pound) flank steak}
			\ingredient{1 pound skinned, boned chicken breast}
			\ingredient{2 red bell peppers,\\ each cut into 12 wedges}
			\ingredient{2 green bell peppers,\\ cut into 16 wedges}
			\ingredient{1 large  Vidalia onion,\\ each cut into 12 wedges}
			\ingredient{cooking spray}
			\ingredient{16 (6 inch) flour tortillas}
			\ingredient{salsa}
			\ingredient{sour cream}
			\ingredient{cilantro}
			\ingredient{cheese}
			\ingredient{guacamole}
		},
	instructions = {
			\instructionsection{Preparing the Marinade}
			\instruction{Combine first 10 ingredients in a large bowl; set aside.}
			\instructionsection{Fajitas}
			\instruction{To prepare fajitas, trim fat from steak. Score a diamond pattern on both sides of the steak. Combine 1 1/2 cups marinade, steak, and chicken in a large zip-top plastic bag. Seal and marinate in refrigerator 4 hours or overnight, turning occasionally. Combine remaining marinade, bell peppers, and onion in a zip-top plastic bag. Seal and marinate in refrigerator for 4 hours or overnight, turning occasionally.}
			\instruction{Prepare grill.}
			\instruction{Remove steak and chicken from bag; discard marinade. Remove vegetables from bag; reserve marinade. Place reserved marinade in a small saucepan; set aside. Place steak, chicken, and vegetables on grill rack coated with cooking spray; cook 8 minutes on each side or until desired degree of doneness.}
			\instruction{Wrap tortillas tightly in foil; place tortilla packet on grill rack the last 2 minutes of grilling time. Bring reserved marinade to a boil. Cut steak and chicken diagonally across the grain into thin slices. Place the steak, chicken, and vegetables on a serving platter; drizzle with reserved marinade.}
			\instruction{Arrange about 1 ounce steak, about 1 ounce chicken, 3 bell pepper wedges, and 1 onion wedge in a tortilla; top with 1 tablespoon salsa, about 1 teaspoon sour cream, and 1/2 tablespoon cilantro.
Fold sides of tortilla over filling. Garnish with cilantro sprigs, if desired. Serve immediately.}
		}
}


%----------------------------------------------------------------------------------------------------------


%----------------------------------------------------------------------------------------
%	RECIPE EXAMPLE - GRILLED SALMON (Columns Layout with Custom Column Ratio)
%----------------------------------------------------------------------------------------

\recipe{%
	image={../images/recipes/grilled-salmon.jpg},
	columnratio={0.35, 0.65},
	title = {Lemon Herb Grilled Salmon},
	indexes = {Lemon Herb Grilled Salmon, Grilled Salmon, Recipes!Entrees, Recipes!Seafood, Salmon, Fish, Lemon, Herbs, Grilled dishes, Healthy recipes, Gluten-free recipes},
	description = {Perfectly grilled salmon with a bright lemon herb marinade. This recipe demonstrates a custom column ratio for the ingredients and instructions layout.},
	serves = {4},
	preptime = {15 mins},
	cookingtime = {12 mins},
	difficulty = {Intermediate},
	origin = {Contemporary},
	tags = {Seafood, Grilled, Healthy, Gluten-Free},
	ingredients = {
			\ingredientsection{Marinade}
			\ingredient{1/4 cup olive oil}
			\ingredient{3 tbsp fresh lemon juice}
			\ingredient{2 cloves garlic, minced}
			\ingredient{1 tbsp fresh dill, chopped}
			\ingredient{1 tbsp fresh parsley, chopped}
			\ingredient{1 tsp lemon zest}
			\ingredient{Salt and pepper}

			\ingredientsection{Salmon}
			\ingredient{4 salmon fillets (6 oz each)\note{For best results, use fillets of uniform thickness so they cook evenly.}}
			\ingredient{Lemon wedges for serving}
			\ingredient{Fresh herbs for garnish}
		},
	instructions ={
			\instruction{In a small bowl, whisk together olive oil, lemon juice, garlic, dill, parsley, lemon zest, salt, and pepper.}
			\instruction{Place salmon fillets in a shallow dish and pour marinade over them. Turn to coat. Cover and refrigerate for 15-30 minutes.}
			\instruction{Preheat grill to medium-high heat (about 375-400°F). Clean and oil the grill grates.}
			\instruction{Remove salmon from marinade and pat dry with paper towels. Discard excess marinade.}
			\instruction{Place salmon skin-side down on the grill. Close the lid and cook for 6-8 minutes without moving.\note{Don't flip the salmon too early—wait until it naturally releases from the grill grates.}}
			\instruction{Carefully flip the salmon using a wide spatula. Cook for another 3-4 minutes until the fish is opaque and flakes easily.\note{The fish should be opaque and flake easily when done.}}
			\instruction{Remove from grill and let rest for 2 minutes. Serve with lemon wedges and garnish with fresh herbs.}
		}
}


%----------------------------------------------------------------------------------------------------------


%----------------------------------------------------------------------------------------
%	RECIPE EXAMPLE - VEGETARIAN CURRY (Simple Layout, Vegetarian)
%----------------------------------------------------------------------------------------

\recipe{%
	layout={simple},
	imageheight={0.25\paperheight},
	imageoverlayspace={0.2\paperheight},
	title = {Thai Red Curry with Vegetables},
	indexes = {Thai Red Curry with Vegetables, Thai Red Curry, Curry, Recipes!Entrees, Recipes!Vegetarian, Vegetables, Coconut milk, Thai cuisine, Spicy recipes, Vegan recipes},
	description = {A vibrant and aromatic vegetarian curry packed with colorful vegetables and creamy coconut milk. Customize with your favorite vegetables.},
	serves = {6},
	preptime = {20 mins},
	cookingtime = {25 mins},
	difficulty = {Intermediate},
	origin = {Thailand},
	tags = {Thai, Spicy, Curry, Vegan, Vegetarian},
	vegetarian = {yes},
	spicy = {yes},
	ingredients = {
			\ingredientsection{Curry}
			\ingredient{2 tbsp coconut oil}
			\ingredient{3-4 tbsp Thai red curry paste\note{Thai red curry paste varies in heat level between brands. Start with less and add more to taste.}}
			\ingredient{1 can (14 oz) coconut milk}
			\ingredient{1 cup vegetable broth}
			\ingredient{2 tbsp soy sauce or tamari}
			\ingredient{1 tbsp brown sugar or maple syrup}
			\ingredient{2 bell peppers, sliced}
			\ingredient{1 medium eggplant, cubed}
			\ingredient{1 cup green beans, trimmed}
			\ingredient{1 cup bamboo shoots}
			\ingredient{1 cup Thai basil leaves}

			\ingredientsection{Serving}
			\ingredient{Cooked jasmine rice\note{Serve over jasmine rice or rice noodles for a complete meal.}}
			\ingredient{Fresh lime wedges}
			\ingredient{Fresh cilantro}
			\ingredient{Sliced red chili (optional)}
		},
		instructions ={
			\instruction{Heat coconut oil in a large pot or wok over medium heat. Add curry paste and cook for 1-2 minutes, stirring constantly, until fragrant.}
			\instruction{Pour in coconut milk and stir well to dissolve the curry paste. Bring to a gentle simmer.}
			\instruction{Add vegetable broth, soy sauce, and brown sugar. Stir to combine.}
			\instruction{Add eggplant and cook for 5 minutes, then add bell peppers and green beans. Simmer for 10-12 minutes until vegetables are tender but still crisp.}
			\instruction{Stir in bamboo shoots and Thai basil. Cook for 2 more minutes.}
			\instruction{Taste and adjust seasoning with more soy sauce, sugar, or curry paste as needed.}
			\instruction{Serve hot over jasmine rice, garnished with fresh cilantro, lime wedges, and sliced chili if desired.}
		}
}


\makeimagepage{
	image={../images/recipes/thai-red-curry.jpg},
	caption = {Thai Red Curry with Vegetables},
}

%----------------------------------------------------------------------------------------------------------


%----------------------------------------------------------------------------------------
%	DESSERTS CHAPTER PAGE
%----------------------------------------------------------------------------------------

\makechapterpage{
	title={Desserts},
	image={../images/book/dessert.jpg},
	layout={right}
}


%----------------------------------------------------------------------------------------------------------


%----------------------------------------------------------------------------------------
%	RECIPE EXAMPLE - CHOCOLATE CAKE (Columns Layout)
%----------------------------------------------------------------------------------------

\recipe{%
	image={../images/recipes/chocolate-cake.jpg},
	imageheight={0.5\paperheight},
	imageoverlayspace={0.45\paperheight},
	title = {Decadent Chocolate Cake},
	indexes = {Decadent Chocolate Cake, Chocolate Cake, Cake, Recipes!Desserts, Chocolate, Cocoa, Buttercream, American cuisine, Celebration cakes, Baked dishes},
	description = {A rich, moist chocolate layer cake with smooth chocolate buttercream frosting. This showstopper is perfect for birthdays and special celebrations.},
	serves = {12},
	preptime = {30 mins},
	cookingtime = {35 mins},
	difficulty = {Advanced},
	origin = {USA},
	tags = {Dessert, Chocolate, Cake, Celebration, Make-Ahead},
	ingredients = {
			\ingredientsection{Cake}
			\ingredient{2 cups all-purpose flour}
			\ingredient{2 cups granulated sugar}
			\ingredient{3/4 cup unsweetened cocoa powder}
			\ingredient{2 tsp baking soda}
			\ingredient{1 tsp baking powder}
			\ingredient{1 tsp salt}
			\ingredient{2 large eggs\note{For best results, bring all ingredients to room temperature before starting.}}
			\ingredient{1 cup buttermilk}
			\ingredient{1 cup strong black coffee, hot}
			\ingredient{1/2 cup vegetable oil}
			\ingredient{2 tsp vanilla extract}

			\ingredientsection{Frosting}
			\ingredient{1 cup (2 sticks) unsalted butter, softened\note{The cake layers can be made a day ahead, wrapped in plastic wrap, and stored at room temperature.}}
			\ingredient{3 1/2 cups powdered sugar}
			\ingredient{1/2 cup unsweetened cocoa powder}
			\ingredient{1/2 cup heavy cream}
			\ingredient{2 tsp vanilla extract}
			\ingredient{1/4 tsp salt}
		},
	instructions ={
			\instructionsection{Prepare the Cake}
			\instruction{Preheat oven to 350°F (175°C). Grease and flour two 9-inch round cake pans.}
			\instruction{In a large bowl, whisk together flour, sugar, cocoa powder, baking soda, baking powder, and salt.}
			\instruction{Add eggs, buttermilk, hot coffee, oil, and vanilla extract. Beat with an electric mixer on medium speed for 2 minutes. The batter will be thin.}
			\instruction{Pour batter evenly into prepared pans. Bake for 30-35 minutes until a toothpick inserted in the center comes out clean.}
			\instruction{Cool in pans for 10 minutes, then turn out onto wire racks to cool completely.}

			\instructionsection{Make the Frosting}
			\instruction{Beat butter with an electric mixer until creamy and smooth, about 2 minutes.}
			\instruction{Sift together powdered sugar and cocoa powder. Gradually add to butter, alternating with heavy cream.}
			\instruction{Add vanilla extract and salt. Beat on high speed for 3-4 minutes until light and fluffy.}

			\instructionsection{Assemble}
			\instruction{Place one cake layer on a serving plate. Spread with about 1 cup of frosting.}
			\instruction{Top with second cake layer. Frost the top and sides of the entire cake with remaining frosting.}
			\instruction{Refrigerate for at least 30 minutes before slicing. Bring to room temperature before serving.}
		}
}


%----------------------------------------------------------------------------------------------------------


%----------------------------------------------------------------------------------------
%	RECIPE EXAMPLE - TIRAMISU (Simple Layout, No Image)
%----------------------------------------------------------------------------------------

\makeimagepage{
	image={../images/recipes/tiramisu.jpg},
	caption = {Classic Tiramisu}
}

\recipe{%
	layout={simple},
	title = {Classic Tiramisu},
	indexes = {Classic Tiramisu, Tiramisu, Recipes!Desserts, Coffee, Mascarpone, Ladyfingers, Italian cuisine, No-bake desserts, Make-ahead meals},
	description = {The beloved Italian dessert with layers of coffee-soaked ladyfingers and creamy mascarpone. A make-ahead dessert that gets better as it sits.},
	serves = {8-10},
	preptime = {30 mins},
	difficulty = {Intermediate},
	origin = {Italy},
	tags = {Italian, Dessert, Coffee, No-Bake, Make-Ahead},
	extrainstructioninfo = {Tiramisu must be refrigerated for at least 4 hours, but overnight is best. This allows the flavors to meld and the ladyfingers to soften perfectly. Use high-quality espresso or strong coffee for the best flavor.},
	ingredients = {
			\ingredientsection{Cream Layer}
			\ingredient{6 large egg yolks}
			\ingredient{3/4 cup granulated sugar}
			\ingredient{1 1/3 cups mascarpone cheese, room temperature}
			\ingredient{2 cups heavy cream, cold}

			\ingredientsection{Coffee Layer}
			\ingredient{2 cups strong espresso or coffee, cooled\note{Use high-quality espresso or strong coffee for the best flavor.}}
			\ingredient{3 tbsp coffee liqueur (optional)}
			\ingredient{40-48 ladyfinger cookies (savoiardi)}

			\ingredientsection{Topping}
			\ingredient{2 tbsp unsweetened cocoa powder}
			\ingredient{Dark chocolate shavings (optional)}
		},
	instructions ={
			\instruction{Whisk egg yolks and sugar in a heatproof bowl. Place over a pot of simmering water (double boiler method) and whisk constantly for 8-10 minutes until thick and pale. Remove from heat and let cool slightly.}
			\instruction{Add mascarpone to the egg mixture and whisk until smooth and well combined. Set aside.}
			\instruction{In a separate bowl, whip heavy cream until stiff peaks form. Gently fold the whipped cream into the mascarpone mixture in three additions.}
			\instruction{Combine cooled espresso with coffee liqueur (if using) in a shallow dish.}
			\instruction{Quickly dip each ladyfinger into the coffee mixture (about 2 seconds per side) and arrange in a single layer in a 9x13 inch dish.}
			\instruction{Spread half of the mascarpone cream over the ladyfingers. Repeat with another layer of dipped ladyfingers and remaining cream.}
			\instruction{Cover with plastic wrap and refrigerate for at least 4 hours or overnight.\note{Overnight refrigeration is best as it allows the flavors to meld and the ladyfingers to soften perfectly.}}
			\instruction{Before serving, dust generously with cocoa powder and garnish with chocolate shavings if desired.}
		}
}


%----------------------------------------------------------------------------------------------------------


%----------------------------------------------------------------------------------------
%	RECIPE EXAMPLE - APPLE PIE (Columns Layout with Image, Multiple Sections)
%----------------------------------------------------------------------------------------

\recipe{%
	title = {Classic Apple Pie},
	indexes = {Classic Apple Pie, Apple Pie, Pie, Recipes!Desserts, Apples, Cinnamon, Pastry, American cuisine, Fall recipes, Baked dishes},
	description = {A traditional American apple pie with flaky, buttery crust and perfectly spiced apple filling. This is comfort food at its finest.},
	serves = {8},
	preptime = {45 mins},
	cookingtime = {55 mins},
	difficulty = {Advanced},
	origin = {USA},
	tags = {Dessert, Pie, American, Fall, Baked, Make-Ahead},
	ingredients = {
			\ingredientsection{Pie Crust}
			\ingredient{2 1/2 cups all-purpose flour}
			\ingredient{1 tsp salt}
			\ingredient{1 tbsp granulated sugar}
			\ingredient{1 cup (2 sticks) cold unsalted butter, cubed\note{For the flakiest crust, make sure all ingredients are cold and handle the dough as little as possible.}}
			\ingredient{6-8 tbsp ice water}

			\ingredientsection{Apple Filling}
			\ingredient{6-7 cups sliced apples (Granny Smith and Honeycrisp mix)}
			\ingredient{2/3 cup granulated sugar}
			\ingredient{1/4 cup all-purpose flour}
			\ingredient{1 tsp ground cinnamon}
			\ingredient{1/4 tsp ground nutmeg}
			\ingredient{1/4 tsp salt}
			\ingredient{2 tbsp lemon juice}
			\ingredient{2 tbsp butter, cut into small pieces}

			\ingredientsection{Topping}
			\ingredient{1 egg, beaten (for egg wash)}
			\ingredient{1 tbsp coarse sugar}
		},
	instructions ={
			\instructionsection{Make the Crust}
			\instruction{Pulse flour, salt, and sugar in a food processor. Add cold butter and pulse until mixture resembles coarse crumbs.}
			\instruction{Add ice water 1 tablespoon at a time, pulsing after each addition, until dough just comes together.}
			\instruction{Divide dough in half, shape into disks, wrap in plastic, and refrigerate for at least 1 hour.}

			\instructionsection{Prepare the Filling}
			\instruction{Preheat oven to 425°F (220°C). In a large bowl, toss sliced apples with sugar, flour, cinnamon, nutmeg, salt, and lemon juice.}

			\instructionsection{Assemble and Bake}
			\instruction{Roll out one disk of dough on a floured surface to 12 inches diameter. Transfer to a 9-inch pie plate.}
			\instruction{Pour apple filling into crust and dot with butter pieces. Roll out second disk of dough and place over filling. Trim excess and crimp edges to seal.}
			\instruction{Cut several slits in top crust for venting. Brush with egg wash and sprinkle with coarse sugar.}
			\instruction{Place pie on a baking sheet. Bake for 20 minutes, then reduce heat to 375°F (190°C) and bake for 35-40 minutes more until crust is golden and filling is bubbling.}
			\instruction{Cool on a wire rack for at least 2 hours before slicing. Serve warm or at room temperature with vanilla ice cream.\note{The pie can be made a day ahead and reheated before serving.}}
		}
}


\makeimagepage{
	image={../images/recipes/apple-pie.jpg},
	caption = {Classic Apple Pie}
}

%----------------------------------------------------------------------------------------------------------

%----------------------------------------------------------------------------------------
%	CONVERSION TABLE
%----------------------------------------------------------------------------------------

\makeconversionpage{
	title={Conversion Tables}
}

%----------------------------------------------------------------------------------------
%	INDEX
%----------------------------------------------------------------------------------------

\printindex

%----------------------------------------------------------------------------------------
%	BACK COVER
%----------------------------------------------------------------------------------------

\makebackcoverpage{
	topcontent={
		{\fontsize{24pt}{28pt}\sourcesanspro\bfseries\selectfont\color{white}\MakeUppercase{About This Book}}\par
		\vspace{0.02\textheight}
		This cookbook represents a collection of cherished recipes passed down through generations, each one telling a story of family gatherings, holiday celebrations, and everyday moments made special by the food we share. From simple comfort foods to elaborate feasts, these recipes have been tested, refined, and perfected over countless meals.\newline\newline
		Whether you're a seasoned cook or just beginning your culinary journey, we hope these recipes bring joy, inspiration, and delicious results to your kitchen.
	},
	image={../images/book/back-cover.jpg},
	imageopacity={0.8},
	imageposition={right},
	columnratio={0.5,0.5},
	verticalsplit={0.5},
	bottomcontent={
		\textbf{From Our Kitchen to Yours:}\par
		\vspace{0.01\textheight}
		Start your morning right with our fluffy \textbf{Banana Pancakes}—a family favorite that's both simple and satisfying. For a special weekend treat, you simply must try our \textbf{Classic French Toast}, golden and perfectly crisp.\newline\newline
		When it comes to main courses, our \textbf{Spaghetti Bolognese} has been a Sunday dinner tradition for decades, while the \textbf{Lemon Herb Grilled Salmon} brings elegance to any weeknight meal. And don't miss our family's favorite dessert—the \textbf{Classic Tiramisu} that has graced countless celebrations and always leaves guests asking for the recipe.\newline\newline
		\textit{Each recipe includes detailed instructions, ingredient lists, and helpful tips to ensure your success in the kitchen.}
	},
	isbn={978-0-123456-78-9},
	publisher={Published by Your Publisher Name},
	copyright={© 2025 All rights reserved.},
	textcolor={white},
	bgcolor={darkgrey},
	divider={true},
	barcodeplaceholder={true}
}

\end{document}
